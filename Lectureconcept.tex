\documentclass[11pt,a4paper]{article}
\usepackage{ls}
\usepackage[english]{babel}

\setcounter{secnumdepth}{-2}

\title{Concept of the Lecture \\[1em] \emph{Modelling Sustainable Systems and
    Semantic Web} \\[1em] Summer Term 2022}

\author{Hans-Gert Gr\"abe}

\date{March 31, 2022}

\begin{document}
\maketitle

\subsection{General}

The lecture is planned to take place in presence (Thursdays 9-11 am., room SG
3-15) and is based on the Flipped Classroom concept. The lecture consists of
three parts.

In the first part we explore the concept of a \emph{technical system} and
introduce the main concepts of TRIZ as an important systematic innovation
methodology.  In contrast to other creativity and innovation methodologies,
TRIZ focuses on the systematisation of engineering experiences.

In the second part, we study more closely aspects of the creation of
conceptual networks for data models on the basis of the \emph{Resource
  Description Frameworks} (RDF), the \emph{Linked Open Data Cloud}, the
emerging \emph{Giant Global Graph} and the importance of these developments
for the organisation of contexts of cooperative action and hence management of
sustainable systems.

Finally, in the third part, we explore the role of data and information and
the generation of new language tools for the development of technical systems
in the context of a civil society and, in particular, the importance of
concept formation processes in cooperative action.

In addition to the general bibliography, each lecture will be accompanied by
literature for preparation which should be \textbf{studied before the
  lecture}, in order to be able to follow the explanations. In the lecture the
topics are presented only cursorily, but there is room to ask questions about
the literature and to discuss individual aspects.

Most of the material is easily found on the internet. Nevertheless we will not
dispense on classical printed literature and your ability to access it.

To support our opening for an international audience, we switch step by step
to English as operational language. The lecture will use a mixed form, with
English slides and German presentation.

The progress of the lecture will be reported regularly in the github
repository \url{https://github.com/wumm-project/Seminar-S22}. There you also
find the schedule and the slides of the individual lectures.

\subsection{Topics of the Lecture}

\begin{itemize}
\item Introduction. Concept of technology
\item Systems and their Development
\item Systems and Sustainability
\item Systemic Structures and Spaces of Action
\item Digital Spaces of Action
\item Internet Basics
\item RDF Basics
\item Modelling Conceptual Worlds
\item The Giant Global Graph
\item Data and Information
\item Storytelling, Conceptualisation, Information and Language
\item Knowledge and Action
\item On a Theory of Cooperative Action
\item Economic Network Structures and "Platform Capitalism"
\item The History of the .NET Project
\item Formation of an Open Culture
\end{itemize}

\subsection{Digital Privacy}

We follow not only a theoretical but also a practical Open Culture approach
and make course materials publicly available.  This also applies to the
(annotated) chat recordings of the lecture, in which your names are mentioned.
We assume your consent to this procedure, if you do not explicitly object.
The discussions themselves will \textbf{not} be recorded.

\end{document}
