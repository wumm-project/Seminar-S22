
\documentclass{article}
\usepackage[utf8]{inputenc}

\title{Contradiction Elimination \\ \large Darrell Mann: Hands-On
Systematic
Innovation
for Business and Management \\ \large Chapter \textit{Problem Solving Tools: Contradictions}, p. 337-352}
\author{Handout by Immanuel von Detten}
\date{June 13, 2022}

\begin{document}

\maketitle

\section{Context}

The context of this chapter is to find strategies to develop a solution for a problem. While the earlier section of the book covers the process to define a problem in a meaningful and useful way, this section of the book is all about the problem solving tools. This handout focuses on the chapter "Problem Solving Tools: Contradictions". The previous chapter "Conflict and Trade-Off Elimination/Inventive Principles" talks in length about how conflicts can be resolved and the respective strategies. Throughout the chapter about contradictions Mann often refers to the chapter about conflict resolution, because in his understanding conflict and contradiction are two ways of representing a problem.

\section{Understanding Conflicts And Contradictions}

Mann describes a framework which helps to find fitting strategies when solving problems. The framework considers conflicts and contradictions. While conflicts are described as a problem where two parameters are in conflict with each other and an optimal solution should be found, contradictions are described as a problem where one parameter is desired to be maximal and minimal, or an entity which should be existing and absent, at the same time.

In this framework, both conflict parameters and the single parameter relevant for the contradiction are linked by an logical "AND". We want conflict parameter 1 to be optimal and we want conflict parameter 2 to be optimal - for contradictions, we want the parameter to be optimal and minimal at the same time. Both ways of looking at a problem strive for an successful outcome.

According to Mann, this structure allows one to transform conflicts into contradictions and vice versa. In a situation where there is an apparent conflict, one should think about reframing the conflict as a contradiction and finding the single shared parameter. The benefit of this approach is that different strategies to find a solution can be applied when the problem is perceived from a different angle.

\section{Strategies To Eliminate Contradictions}

Mann develops the following strategies to eliminate contradictions:

\begin{enumerate}
    \item Separation in space - \textit{where?}
    \item Separation in time - \textit{when?}
    \item Separation on condition - \textit{if?}
    \item Separation by transition to an alternative system
\end{enumerate}

The list is hierarchical. When looking for strategies, one should follow the list top-to-bottom and prioritize separation in space over time, time over condition and so forth. 

By answering the questions for the identified parameter, it becomes possible to pick strategies from the list of "Inventive Principles" (p.316-336). The list contains helpful input to tackle a problem, e.g. when brainstorming, and supplements them with concrete tips and examples. The list is compiled by looking at examples from business. As an illustration of the list's structure, the first three principles are:
\begin{enumerate}
    \item Segmentation (p.316)
       \begin{enumerate}
            \item Divide a system or object into independent parts.
            \item Make a system or object easy to disassemble.
            \item Increase the degree of fragmentation/segmentation.
       \end{enumerate}
    \item Taking out/separation (p.317)
       \begin{enumerate}
            \item Separate an interfering part or property from a system or object, or single out the only necessary part (or property).
       \end{enumerate}
    \item Local quality (p.318)
       \begin{enumerate}
            \item Change the structure of an object or system from uniform to non-uniform, change an external environment (or external influence) from uniform to non-uniform.
            \item Make each part of an object or system function in conditions most suitable for its operation.
            \item Make each part of an object or system fulfill a different and useful function.
       \end{enumerate}
\end{enumerate}

Mann combines the four strategies for contradiction with the list of Inventive Principles to create a list of most relevant/most used inventive principles for each of the four strategies (referred to as \textit{contradiction solution route}), basing the order of the list on case studies again.

Mann stresses the benefit of the process of resolving contradictions over the process of optimizing. According to him, resolving the contradiction instead of optimizing can lead to a situation where the connection between a parameter and the successful outcome can be resolved. For him, optimizing is not a very creative process, but a process focused on measurable success. In juxtaposition, resolving the contradiction needs creativity to think about new ways of designing the system - the described framework aims to create a systematic approach to this creative process. 

\section{Perception Of Contradictions}

In the last section of the chapter, Mann talks about how important it is to differentiate between real and perceived contradictions. His differentiation is based on the assumption that people can have different perceptions of reality and there is an "actual physical reality" (p.350). The difference of the mental "map", which represents perception, of the "territory", which represents facts, needs to be looked at: Which person is actually mapping the "territory", so which perception is closer to reality? Afterwards the situation can be analyzed and concluded if the contradiction is real or perceived. His term for real contradictions are "right-versus-right" situations and for perceived contradictions "right-versus-wrong". Perceived problems need to be handled differently than real problems, as the entity with the wrong perception needs to be urged to reconsider their point of view. To do this the previously described framework of understanding conflicts/contradictions as connected by a logical "AND" needs to be broken up: The link of the "AND" between conflict parameters/contradiction parameter needs to be broken and the . For "right-versus-right" situations the same process as described in the previous section can be applied.

\section{Summary}

Mann summarizes the process to resolve contradictions as following:

\begin{enumerate}
    \item Identify a contradiction
    \item Work through the first three pairs of separation questions - where, when and if
    \item If applicable, use multiple Inventive Principle triggers in combination
    \item If no separation strategies are possible or other solutions are desired, use the fourth strategy of "alternative ways"
    \item Apply the suggestions of the Inventive Principles and develop possible solutions
\end{enumerate}

\end{document}
