\chapter{Case Studies}

Darrell Mann takes a closer look at case studies to analyze measurement problems in the real world.
In this essay we only want to discuss two of those case studies.
One regarding the photographic paper industry, to deepen our understanding of what to measure.
The other regarding project management in general with a closer look on how to measure resources.

\section{Measuring Business Performance}

The digilization has led to the use of less photo-paper.
Digital cameras can delete bad photos instead of printing them; pictures get stored on hard drives, and the film industry has become digital.
Due to this, the photo-paper industry is in front of a conflict.
Less paper will be needed in the future, and the decline of the industry is certain.

Now it is interesting what the industry normally measures to maintain control.
Mostly variables that are easy to measure: sold amount of paper, revenue, costs.
In the day to day business, these measurements are very helpful in adjusting the input and output of the company.
But for the long run looking only at those information won't help with the declining need for photopaper.
We need to take a step back and consider the companies main useful function.
Photo-paper used to be a mean to be able to see photos and to keep and store them.
The function to be able to see them is now obsolete, but the memory function of pictures remains the same.
People love photo albums but it is hard work to organize them.
The industry must realize that their product is mostly used to store memories and be able to revisit them.
This opens up a whole new palette of options.
An online platform could be introduced to store photos and share them with friends and family.
With this in mind, the dimensions that should be measured change completely.
Aspects like most views, most shares, group size of viewers and number of images get much more important.
These are future oriented measurements, but of course they are much harder to measure.

More than ten years have already passed since the publication of Manns book.
We can use this as an advantage to have a look at today's memory business.
Many companies have joined the memory business with similar ideas as proposed in Darrell Manns book.
Apple creates photo albums for you by using the time and location information of the photos.
Different social media platforms create highlight reels of the last year.
Still-standing photos got upgraded to shorter, more vivid video clips.
Even discounter offer to print calendars, photo albums and other merchandise out of your favorite pictures.
This shows that Mann was right with his suggestion to measure the memory function of products, since this function is very much alive.

\section{Project Management System Measurements}

Project management seems to be inherently flawed.
Of all projects, 85\% end either to late, over-budget or under the required specification, sometimes even as a combination of all three.
Many project managers now use a dashboard, similar to that of a car, to have a broad overview of all data.
However the data is organized, it is vital that the data is relevant, acquireable and accurate.
One of the main problems in project management comes out of data acquisition.
If there is not enough data to understand a system, often more project managers get hired to gain more data.
We have seen that this increases the complexity of a system and should be avoided.
Instead, lets apply the first strategy for how to measure: modify the system so that there is no need to make the detection or measurement.
The ideal data acquisition should happen out of existing resources.
Popular systems to get this data would be digital diary and accounting systems, automated reporting, suggestion schemes, or the amount of phone calls.
These are the first sources that should fill a project managers dashboard.
It is so important not to first introduce more project managers because it can lead to individuals losing their sense of responsibility.
If someone else, here the project manager or a quality department, is there to maintain a timeline or standard, the individual could stop caring for being responsible for that measurement themselves, since there is someoone whose job is exactly that.



















