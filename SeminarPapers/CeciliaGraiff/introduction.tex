\chapter{Knowledge}
In today's world, easy access to knowledge sources and databases is granted to everyone. However, the huge amount of data can also be confusing. The strategies for the access to knowledge therefore often rely on the idea that a similar problem was already solved by someone else. This requires an abstraction from the specific problem onto a more general level. However, this approach is not always reliable or even possible. In many cases, a specific strategy for knowledge access and management is necessary.\\ \\  In this paper, the steps to the access and management of knowledge will be handled according to Darrell Mann's theories, explicated in his book \textit{Hands-On Systematic Innovation for Business and Management}. \cite{darrell2004hands}
This paper will follow the structure of Mann's chapter on Knowledge and present the author's thesis. Most importantly, it will try to address open questions and critical aspects of Mann's work and analyze them by comparing them with other sources.\\
\\
The second chapter will handle problems and strategies for knowledge access. Systematic innovation processes rely on a subject-action-object triple. Of this triple, Mann chooses to focus on the actions as \textit{functions}. This function-based approach, considered both as a search and as a knowledge organization strategy, will be presented, analyzed and compared to the well-known Entity-Relationship Model and its application to knowledge databases.\\
\\
Alongside with research strategies, the support of research tools is fundamental for the access to knowledge. Therefore, the third chapter will present different kinds of search tools and focus on the importance of context in the developing and the use of the mentioned search tools.\\
\\
Mann's book also presents a brief \textit{excursus} on knowledge and wisdom. This theme will also be addressed - even though its complexity and multidisciplinarity makes it difficult to reach a complete and not ambiguous definition of both concepts. 
\\ \\ Not only the access to knowledge can be difficult, but also its management, especially in the case of business models. For example, who should take care of knowledge and knowledge-related problem in a business company? When should the acquired knowledge be forgotten? The questions of knowledge management are many. The last chapter of this book will try to sum up Mann's point of view on knowledge management in business contexts.
