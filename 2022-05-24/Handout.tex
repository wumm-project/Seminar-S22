\documentclass[12pt,a4paper]{article}
\usepackage[utf8]{inputenc}
\usepackage[T1]{fontenc}
\usepackage[english]{babel}
\usepackage{parskip}
\usepackage{amsmath}
\usepackage[top=4cm]{geometry}

\author{Victor Jüttner}
\title{Problem Solving Tools \\ \textbf{Measurement Problems} \\ \vspace{5mm} \large Handout}


\begin{document}
\maketitle


\section{What is Measurement?}
When one talks about measurement without context, measuring lengths, weight or temperature comes to mind. These everyday measurements are often very concrete. Lets take the example of baking a cake. To achieve the desired result we weigh the different ingredients on a scale. Intuitively one knows why measuring, here weighing, is necessary. No one in there right mind would take unnecessary measurements like the circumference of the eggs nor be in doubt about the proper tool for weighing flour. Now if we switch the context to the abstract world of business, where measuring is just as important, these questions are much harder to answer.
Here problems like missing measuring tools or doubt about the correct measurements do arise. This presentation will show Darrel Mann's recipe to recognize and solve those problems.


\section{What To Measure and Why To Measure It?}
To know what we should measure, we analyze the ideality equation (1) that is used throughout the systematic innovation methodology.

\begin{equation}
	 \text{Ideality} = \text{Perceived} \left(\frac{\text{Benefits}}{\text{Cost} + \text{Harm}}\right)
\end{equation}

Cost is easily measured and receives a lot of attention. Benefits means the value of the product or service from the point of view of the receiver and harm combines waste, environmental awareness and sustainability issues. These three variables are mostly measured with monetary methods. Perceived respects the intangibles of a model and deals with its complexity and uncertainty.

With these four variables we measure the overall ideality of a system, in our case often a company. Without the measurement of ideality no system would be viable. In a viable system management has control over the staff, the product or service and to some extend to the customer. This control flow is identical with the parts of the system that needs to be measured for feedback.

To understand a viable system we need to measure it. This need for measurement introduces complexity in a system that increases up to a point where the system needs to be changed or it will no longer be viable. Change needs to be in the form of integrating or eliminating measurements to decrease the complexity. Increasing and decreasing complexity of a system form a cycle.


\section{How to Measure}
The following list a top-down approach to effective measuring strategies. The strategies from the top of the list will yield more ideal solutions than the ones from the bottom.

\begin{itemize}
	\item[a)] Modify the system so that there is no need to make the detection or measurement
	\item[b)] Make the detection or measurement on a copy, image or replica of the object or system
	\item[c)] Transform the problem into one involving successive measurement of changes
	\item[d)] Add a new element (communication or person or element) to provide an easily detectable parameter related to the parameter required to be measured or detected
	\item[e)] If it is not possible to modify the system, then introduce an easily detected element to the surrounding environment
	\item[f)] If it is not possible to introduce an easily detectable element into the environment surrounding a system, obtain the desired measurement by detecting changes in something already in the environment
	\item[g)] Make use of psychological effects to help make the measurement
	\item[h)] Use emotional effects to help to make the measurement
	\item[i)] Use the inverse or opposite system to make the measurement
\end{itemize}


\end{document}