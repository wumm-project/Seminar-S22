\documentclass[11pt,a4paper]{article}
\usepackage{ls}
\usepackage[english]{babel}

%\setcounter{secnumdepth}{-2}

\title{Concept for the Research Seminar\\ "Systematic Innovation Methodology"}

\author{Hans-Gert Gr\"abe, Ken Pierre Kleemann, Simon Johanning,\\ Nadine
  Schumann (Uni Leipzig), Ralf Laue (WH Zwickau) }

\date{April 5, 2022}

\begin{document}
\maketitle

\section{Aim and Methodology of the Seminar}

The concept of a \emph{system} plays a prominent role in computer science when
it comes to database systems, software systems, hardware systems, accounting
systems, access systems, etc.  In general, computer science is regarded by a
majority as the "science of the \emph{systematic} representation, storage,
processing and transmission of information, especially their automatic
processing using digital computers" (German Wikipedia).  Also certain relevant
professions such as the \emph{system architect} are in high esteem by IT
users.

However, the significance of the concept of system extends far beyond the
field of computer science -- it is fundamental for all engineering sciences
and as \emph{Systems Engineering} with the ISO/IEC/IEEE-15288 standard
"Systems and Software Engineering", it is also the subject of international
standardisation processes.  Even more, the concept of systems also plays an
important role in the description of complex natural and cultural processes.

In contrast to an \emph{artefactual} dimension of computers and software
running on them, Systems Engineering is more concerned with the \emph{design
  and use} of such systems as a result of \emph{cooperative action}.  The
English Wikipedia emphasises 
\begin{quote}
  "Systems Engineering is an interdisciplinary field of engineering and
  engineering management that focuses on how to design, integrate, and manage
  complex systems over their life cycles. At its core, systems engineering
  utilises systems thinking principles to organise this body of knowledge. The
  individual outcome of such efforts, an engineered system, can be defined as
  a combination of components that work in synergy to collectively perform a
  useful function."
\end{quote}

Ian Sommerville takes a similar view when he identifies the design and
operation of socio-technical systems as the focus of his textbook
\emph{Software Engineering}.
\begin{quote}  
  \emph{Socio-technical systems} contain one or more technical systems, but
  beyond that – and this is crucial – the knowledge of how the system should
  be used to achieve a broader purpose. This means that these systems have
  \emph{defined work processes}, \emph{human operators} as integral part of
  the system, are \emph{governed by organisational policies} and are affected
  by \emph{external constraints} such as national laws and regulations.
  \cite[p. 48]{Sommerville}
\end{quote}

\emph{Systematic Innovation Methodology} takes these concepts of
organisational computer science\footnote{The subject of computer science is
  not only hardware and software, but also orgware.} and combines them with
questions of Business Process Modelling. In such practices, the worlds of
experience of the engineer and the manager meet, at least insofar as the
latter applies \emph{scientifically based methodologies} to organise his work.
However, structured approaches on the basis of systemic modelling is
increasingly shaping managerial action as well, making the latter an
\emph{engineer of social relations} who has to solve essential tasks in the
design of more comprehensive socio-technical systems such as organising
companies or inter-company supply chain structures.

\textbf{In the seminar}, we want to learn more about such modern management
appoaches in which \emph{common conceptualisations} and
\emph{consensus-oriented decision-making processes} are central and of crucial
importance for the success and ways of formation and consolidation of new
systemic structures.  We are particularly interested in the connection between
the dialectical resolution of contradictory requirement situations in the
sense of \emph{TRIZ methodology}, the transition of such approaches to
\emph{Business TRIZ}, and the emergence of common conceptual and notational
worlds as a result of the application of suitable Semantic Web technologies.
A special emphasis will be put on the work of the \emph{Methodological School
  of Management} and the Moscow Methodological Circle around
G.P. Shchedrovitsky \cite{MSM}.

\section{Embedding of the Seminar}

The seminar is designed as a permanent \emph{Research Seminar} in which
developments on the topic outlined above have been studied in more detail
since 2019 with different focuses. A summary of findings and discussions is
available as \emph{Seminar Notes:}
\begin{itemize}
\item Seminar on System Theory. Winter Term 2019/20.\\
  \url{https://nbn-resolving.org/urn:nbn:de:bsz:15-qucosa2-748430} 
\item Management Theories. Summer Term 2021.\\
  \url{http://www.informatik.uni-leipzig.de/~graebe/skripte/Notes-S21.pdf}
\item Sustainability, Environment, Management. Winter Term 2021/22.\\
  \url{http://www.informatik.uni-leipzig.de/~graebe/skripte/Notes-W21.pdf}
\end{itemize}

The course offered by the working group \emph{Systematic Innovation
  Methodologies} at the InfAI is part of the WUMM project\footnote{WUMM stands
  in German for \emph{Widersprüche und Managementmethoden} (Contradictions and
  Management Methods).} aims at a better understanding of such management
processes. Our starting point was TRIZ as a systematic innovation methodology
derived from engineering experience in contradictory requirement situations.
With the field of \emph{Business TRIZ}, which has been unfolding for about 20
years, a transfer of experience is being actively promoted, embedded in older
management cultures and theories.  A better understanding of such approaches
to management issues and their connection to systemic concepts and approaches
remains in the focus of our seminar.

In recent years, co-operative action by differently specialised experts has
become increasingly important.  In such interdisciplinary work contexts, the
development of \emph{common conceptual systems} of sufficient performance
proves to be a difficult problem that can be supported by digital semantic
technologies.  Parallel to these challenges \emph{agile approaches} play a
major role, not only in the field of management, but also increasingly in the
solution of socio-technical and engineering problems concerning ongoing
co-operative actions in multi-stakeholder contexts -- for example with the
concept of \emph{technical ecosystems}.

\section{Organisation of the Course}

In the \textbf{seminar} we jointly explore different aspects of systematic
innovation methodologies.  With this seminar, we are approaching comprehensive
topics that are new to us, which offers the opportunity to participate in a
joint academic explorative process on a basis of equals. This bears
opportunities, but also challenges.  The students are expected to actively
participate in the seminar through seminar discussions, presentations and last
but not least by reading the relevant materials.  For the successful
completion of the seminar, a topic has to be presented as discussion leader
and a handout of 2--3 pages on the topic has to be submitted in advance.

The seminar is accompanied by a \textbf{lecture} \emph{Modelling Sustainable
  Systems and Semantic Web} (Thursday 9-11 a.m.) in which important concepts
of our interdisciplinary course programme such as
\begin{itemize}[noitemsep]
\item technology as combination of globally available procedural knowledge,
  institutionalised procedures and private procedural skills, 
\item sustainability requirements and systemic concepts,
\item digital change and concepts of Semantic Web technologies,
\item concept and knowledge formation processes,
\item cooperative action, network economies and Open Culture
\end{itemize}
are developed in more detail. The lecture and the seminar are not directly
related to each other, but conceptual frameworks developed in the lecture will
be heavily present in the seminar. There is a slide
stack\footnote{\url{http://www.informatik.uni-leipzig.de/~graebe/skripte/Folien-W21.zip}}
available from the lecture in the previous semester.

All materials and seminar reports that can be made publicly available, will be
published in the github repository
\url{https://github.com/wumm-project/Seminar-S22}.

\section{Seminar Organisation}

The seminar will be held weekly on Tuesdays 9-11 a.m.  synchronously online.
Prior to each appointment participants have to study the assigned reading to
be in a position to discuss the problems in the seminar.  The seminar is
moderated by a \emph{discussion leader}, who prepares a short workout of 2--3
pages and makes it available to the participants in advance \emph{before the
  seminar} (by Sunday evening).

Students find more about the seminar in the Saxonian e-learning platform
OPAL\footnote{\url{https://bildungsportal.sachsen.de/opal/} -- Course
  S22.BIS.SIM.}.  The platform will be used for organisational purposes only.
The \textbf{primary source for the seminar plan} is the (actual version of
the) file \texttt{Seminarplan.md} in the github repository \emph{Seminar-S22}.

\section{Examination. Topics for Seminar Work}

In order to successfully complete the seminar module, one of the seminars has
to be moderated as discussion leader, for this seminar a short workout has to
be prepared and made available to the participants and a Seminar Paper (about
20 pages) has to be written.

The seminar is graded from the evaluation of this Seminar Paper, which has to
be completed until the end of the semester on September 30, 2022.

\section{Privacy}

We follow an Open Culture approach not only theoretically, but also
practically and make course materials publicly available.  This also applies
to the course materials you have to produce (presentations, seminar papers) as
well as to (annotated) chat sessions of the seminar discussions, in which your
names are also mentioned.  We assume your consent to this procedure if you do
not explicitly object.  The seminar discussions themselves are \textbf{not}
recorded.

To simplify the further use of the materials and texts, the papers are asked
to be compiled in English using {\LaTeX}.  Also the {\LaTeX} source should be
provided under the terms of the
CC-BY\footnote{\url{https://creativecommons.org/licenses/by/4.0/}} license in
order to create a corresponding corpus of texts that can be used to accompany
similar efforts in the OpenDiscovery project. Of course, this cannot be
"decreed". \textbf{Please inform the seminar instructor if you do not wish to
  make your work available for exchange under these conditions}.

\section{Seminar plan}

The seminar starts on April 19, 2022 with a kick-off meeting.  The exact
topics and themes will be published at the beginning of the seminar, when the
number of participants can be estimated more precisely.

A non-exhaustive list of possible topics for student presentations is compiled
in the \emph{Seminar Plan}.

\begin{thebibliography}{xxx}
\bibitem{MSM} Shchedrovitsky, G. P. (2014). Selected Works. A Guide to the
  Methodology of Organisation, Leadership and Management. In: Viktor
  B. Khristenko, Andrei G. Reus, Alexander P. Zinchenko et
  al. (2014). Methodological School of Management. Bloomsbury Publishing.
  ISBN 978-1-4729-1029-5.

  Available as e-book at UB Leipzig\\
  \url{https://ebookcentral.proquest.com/lib/leip/detail.action?docID=6159470}
\bibitem{Sommerville} Sommerville, I. (2007).  \emph{Software Engineering}.
  Cited by the German 8th edition. Pearson Studium.
\end{thebibliography}

\end{document}
