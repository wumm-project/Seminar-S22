\documentclass{article}
\usepackage[utf8]{inputenc}

\title{Problem Solving Tools \\ \textbf{Knowledge} \\ \vspace{5mm} \large Handout}
\author{Cecilia Graiff}
\date{June 28^{th}, 2022}

\begin{document}

\maketitle

\section{Introduction}
In today's world, easy access to knowledge sources and databases is granted to everyone. However, the huge amount of data can also be confusing, and accessing and managing knowledge requires a strategy. Systematic innovation methods are built around a subject-action-object, or, more semantically speaking, noun-verb-noun template. To apply this template, solutions are thought in terms of functions, and the focus thus lies on verbs instead of nouns. 
Apart from knowledge access, also research tools and knowledge management will be handled in this handout, which aims at summarizing Darrell Mann's proposal of a successful business model for dealing with knowledge.

\section{Accessing Knowledge}
Knowledge access is better achieved when solutions are arranged in terms of functions. From the business point of view, this is reflected by a focus on the final aim of the sold product; For example, a business company which sells "cleaned clothes" will probably turn out to be more successful than one that sells cleaning powder. On the technical side, the question that arises is: What are function words? As mentioned above, systematic innovation methods are built around a noun-verb-noun template. Focusing on the function means extracting the action. That's why function words are verbs. They can be further divided into categories, like helping verbs, management verbs, accomplishment verbs, etcetera.

\section{Knowledge Search Tools}
Another fundamental ingredient to Mann's recipe are research tools. They are divided into search engines, semantic search tools, user-defined context search tools, and intelligent agent-based search tools. The first category is based on keyword research, and also considers the distance between words as a parameter. A famous example? Google. The second category, on the other hand, is based on the subject-action-object template: the tool is capable of extracting subjects, actions and objects. The third category also takes into account the context of the search, specified by the user, while the last one learns about this context by observing the user.

\section{Knowledge Management}
The above mentioned ideas and concepts allow the creation of an universally applicable knowledge framework based on the idea that "someone, somewhere has already solved this problem". However, this idea cannot always be applied, and that's where the research strategies come to help. By applying them, however, a successful business model for managing knowledge should remember to trust the power of self-organization systems and not delegate knowledge management to a department, while also forgetting knowledge that is no longer useful.


\end{document}
