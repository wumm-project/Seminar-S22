\documentclass{beamer}
\usepackage{lsfolien,enumitem}
\usepackage[english]{babel}

\myfootline{System Modelling and Semantic Web -- Summer Term 2022}{Hans-Gert
  Gr\"abe}

\newcommand{\ueberschrift}[1]{\begin{center}\bf #1\end{center}}

\title{Modelling Sustainable Systems\\ and Semantic Web\\[6pt] Introduction
  \vskip1em}

\subtitle{Lecture in the Module 10-202-2312\\ for Master Computer Science}

\author{Prof. Dr. Hans-Gert Gräbe\\
\url{http://www.informatik.uni-leipzig.de/~graebe}}

\date{April 2022}
\begin{document}

{\setbeamertemplate{footline}{}
\begin{frame}
  \titlepage
\end{frame}}

\section{Background}
\begin{frame}{Background}

\ueberschrift{Interdisciplinarity}

Guiding motto of the University of Leipzig:
\begin{center}
  A Tradition of Crossing Boundaries\\ (Aus Tradition Grenzen überschreiten) 
\end{center}

\begin{itemize}
\item[$\bullet$]  Boundaries: Humanities -- Science -- Technology
\item[$\bullet$]  Tradition: The Faculty of Philosophy until 1951
\end{itemize}

But what about technology?

Our mode of production today is closely linked to the use of science and
technology. This brings advantages, but also problems.
\end{frame}

\begin{frame}{Background}
This was not always so. This bourgeois (or capitalist) mode of production has
developed over the last 500 years, first in Western Europe and has since
spread around the globe.

Technology and its use have undergone important metamorphoses during the last
200 years, which had also a decisive influence on the organisational forms of
that mode of production, for example with the establishment of the factory
system and industrial production in the second half of the 19th century.
\end{frame}

\begin{frame}{Background}
This is the beginning of what brought Marx to the vision of a society in which
the "social metabolism" ("gesellschaftlicher Stoffwechsel", MEW 23) is
organised in a way that
\begin{quote}\small
  no longer does the worker insert a modified natural thing as middle link
  between the object and himself; rather, he inserts the process of nature,
  transformed into an industrial process, as a means between himself and
  inorganic nature, mastering it. He steps to the side of the production
  process instead of being its chief actor.\vskip1em

  Es ist nicht mehr der Arbeiter, der modifizierten Naturgegenstand als
  Mittelglied zwischen das Objekt und sich einschiebt; sondern den
  Naturprozess, den er in einen industriellen umwandelt, schiebt er als Mittel
  zwischen sich und die unorganische Natur, deren er sich bemeistert. Er tritt
  neben den Produktionsprozeß, statt sein Hauptagent zu sein.  (MEW 42,
  ch. 14)
\end{quote}
\end{frame}

\begin{frame}{Background}
Marx goes on to state that the development of the productive forces is
\emph{necessarily} heading in such a way towards the organisation of the
"social metabolism".
\begin{quote}
  But, once adopted into the production process of capital, the means of
  labour passes through different metamorphoses, whose culmination is the
  \emph{machine}, or rather, an \emph{automatic system of machinery} (system
  of machinery: the \emph{automatic} one is merely its most complete, most
  adequate form, and alone transforms machinery into a system), set in motion
  by an automaton, a moving power that moves itself; this automaton consists
  of numerous mechanical and intellectual organs, so that the workers
  themselves are cast merely as its conscious linkages.\vskip1em

  German translation on the next slide.
\end{quote}
\end{frame}

\begin{frame}{Background}
\begin{quote}
  In den Produktionsprozess des Kapitals aufgenommen, durchläuft das
  Arbeitsmittel aber verschiedene Metamorphosen, deren letzte die Maschine ist
  oder vielmehr ein automatisches System der Maschinerie (System der
  Maschinerie; das automatische ist nur die vollendetste adäquateste Form
  derselben und verwandelt die Maschinerie erst in ein System), in Bewegung
  gesetzt durch einen Automaten, bewegende Kraft, die sich selbst bewegt;
  dieser Automat besteht aus zahlreichen mechanischen und intellektuellen
  Organen, sodass die Arbeiter selbst nur als bewusste Glieder desselben
  bestimmt sind. (MEW 42, ch. 13)
\end{quote}
\end{frame}

\begin{frame}{Background}
The mastery of that "automaton", that apparently "natural" development of
society, is on the agenda today, because its "naturalness" is increasingly
undermining the very conditions of human existence on our planet.

Marx's vision that in this process the worker "steps to the side of the
production process instead of being its chief actor" is based on a very narrow
understanding of "production process".
\end{frame}

\begin{frame}{Background}
In the 150 years since then this narrow understanding has been replaced by a
common modern understanding of "production process", in which
\begin{quote}\small
  it is neither the direct human labour the worker himself performs, nor the
  time during which he works, but rather the appropriation of his own general
  productive power, his understanding of nature and his mastery over it by
  virtue of his presence as a social body – it is, in a word, the development of
  the social individual which appears as the great foundation-stone of
  production and of wealth.\vskip1em

  ... es weder die unmittelbare Arbeit ist, die der Mensch selbst verrichtet,
  noch die Zeit, die er arbeitet, sondern die Aneignung seiner eignen
  allgemeinen Produktivkraft, sein Verständniss der Natur, und die
  Beherrschung derselben durch sein Dasein als Gesellschaftskörper – in einem
  Wort die Entwicklung des gesellschaftlichen Individuums, die als der große
  Grundpfeiler der Produktion und des Reichtums erscheint. (MEW 42, ch. 14)
\end{quote}

\end{frame}

\begin{frame}{Background}
In short, it is a question to overcome the "naturalness" of the "automaton" --
the socio-technical-cultural "apparatus" created by human as social being  --
and put it under the control of the united humanity.

This requires one thing above all -- educated and committed personnel who are
capable of exercising also their civic responsibility (up to Art. 20 of our
constitution).  Academic institutions are requested to deliver an important
contribution to this.  This also requires to cross the old and new boundaries
between Humanities -- Sciences -- Technology.
\end{frame}

\begin{frame}{Background}
The academic education system has been on this path for more than 100 years,
as the development of technological academic educational institutions in
Leipzig shows.
\begin{itemize}
\item[$\bullet$] 1838 Foundation of the Royal Saxonian School of Building
  Professions (K\"oniglich-sächsische Baugewerkeschule) in Leipzig by Albert
  Geutebrück
\item [$\bullet$] 1856 Foundation of the VDI, the \emph{German Association of
  Engineers} 
\item[$\bullet$] 1875 Foundation of the Municipal School of Trades
  (st\"adtische Gewerbeschule) in Leipzig as the historical root for education
  in mechanical and electrical engineering.\medskip

  Realisation that tradesmen need a thorough technical education in addition
  to a general higher education ("humanistische Bildung").
\end{itemize}
\end{frame}

\begin{frame}{Background}
\begin{itemize}
\item[$\bullet$] 1909 Royal Saxonian Building School
\item[$\bullet$] 1914 Technical school for librarians
\item[$\bullet$] 1920 Saxonian State Building School
\item[$\bullet$] 1949 Technical School for Energy Markkleeberg
\item[$\bullet$] 1954 Leipzig College of Civil Engineering
\item[$\bullet$] 1956 Leipzig School of Engineering for Gas Technology
\item[$\bullet$] 1965 School of Engineering for Automation Technology
\item[$\bullet$] 1970 School of Engineering for Energy Management Leipzig
\item[$\bullet$] 1969 Leipzig College of Engineering
\item[$\bullet$] 1977 Unification into Leipzig Technical University
\item[$\bullet$] since 1992 University of Applied Sciences for Technology,
  Economics and Culture
\end{itemize}
\end{frame}

\begin{frame}{Background}
In the 20th century, with \emph{engineers} a whole new professional group
appeared.  Nowadays new professions such as computer scientists, are already
crossing the border between science and technology and can graduate from our
Faculty to obtain a doctorate in science (Dr. rer. nat.), but also a doctorate
in engineering (Dr.-Ing.).  The situation with the Humanities has not yet been
clarified, but at least you can qualify for a Master in Digital Humanities.
\end{frame}

\section{Course Programme}
\begin{frame}{Course Programme}

\ueberschrift{Background and Objectives}

The course is an interdisciplinary offer in the Master in Computer Science and
Master in DH, but can also be taken as a minor (Nebenfach). 

The aim of the course is to illuminate important developments in the outlined
field of development.  
\end{frame}

\begin{frame}{Course Programme}
\ueberschrift{Four Theses}
\begin{itemize}
\item[1)] The short digital age is already over, the corona age started.
\item[2)] Whereas the digital transformation was still characterised by a
  rapidly growing "world of digital data", through the analysis and processing
  of which influence on real-world processes was gained, we are now faced with
  the challenge of using these tools to meet the challenges of the corona
  crisis.
\item[3)] These challenges are only a small foretaste of the challenges that
  climate change will pose.
\item[4)] These challenges are closely linked to fundamental questions not
  only about our economy or mode or production, but also of our understanding
  of technology and science.
\end{itemize}
\end{frame}

\begin{frame}{Course Programme}
It is therefore appropriate to address the three topics
\begin{itemize}
\item[$\bullet$] Social structures of digital change
\item[$\bullet$] Modelling of sustainable systems
\item[$\bullet$] Conceptualisation processes and the Semantic Web
\end{itemize}
and to develop a set of conceptual and terminological tools that are suitable
for a viable analysis of these topics.

The conceptual toolkit to be developed is oriented towards various aspects of
the development of socio-technical systems that are addressed in the lecture
and in the seminar.
\end{frame}

\section{Organisational Matters}
\begin{frame}{Organisational Matters}
The Seminar Module 10-202-2312 (5 CP) "Applied Computer Science" includes
\begin{itemize}
\item[$\bullet$] a lecture "Modelling of Sustainable Systems and Semantic Web"
\item[$\bullet$] a seminar "Sustainability, Environment, Management"
\end{itemize}
Examination:
\begin{itemize}
\item[$\bullet$] \textbf{Prerequisite for examination:} successfully completed
  seminar.
\item[$\bullet$] \textbf{Examination:} Seminar paper.
\end{itemize}
\end{frame}

\begin{frame}{Organisational Matters}
More about this in OPAL \url{https://bildungsportal.sachsen.de/opal} in the
course S22.BIS.SIM.  There, please enrol first in the course and then in the
corresponding group.

You can access OPAL with the data of your studserv account.

You will find a more detailed lecture concept in the github repo
\url{https://github.com/wumm-project/Seminar-S22}.
\end{frame}

\begin{frame}{Data protection}

We follow an Open Culture approach not only theoretically but also practically
and make course materials publicly available. This also applies to the course
materials you have to produce (presentations, handouts) as well as to
(annotated) chat sessions of the seminar discussions, in which your names are
also mentioned. \textbf{We assume your consent to this procedure if you do not
  explicitly object}. The discussions themselves are not recorded.

\end{frame}

\begin{frame}{Organisational Matters}

\begin{itemize}
\item[$\bullet$] Lecture: Thursdays 9:15-10:45, SG 3-15
\item[$\bullet$] The Flipped Classroom Concept
\item[$\bullet$] Continuously updated lecture plan and list of references in
  the \texttt{Lecture/README.md} file in the github Repo.  
\item[$\bullet$] Further (mainly organisational) information also in the forum
  of the OPAL course.
\item[$\bullet$] Seminar: Tuesdays 9:15-10:45, starts on April 19, synchronous
  digital in the BBB room BIS.SIM,
  \url{https://meet.uni-leipzig.de/b/gra-w2c-fhz-qnp}
\end{itemize}
\begin{center}\LARGE\bf
  Questions ?
\end{center}

\end{frame}

\end{document}
