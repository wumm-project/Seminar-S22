\documentclass{beamer}
\usepackage{lsfolien}
\usepackage[english]{babel}
\usepackage[utf8]{inputenc}

\myfootline{System Modelling and Semantic Web -- Winter term
  2021/22}{Hans-Gert Gräbe}

\newcommand{\ueberschrift}[1]{\begin{center}\bf #1\end{center}}

\title{Modelling Sustainable Systems\\ and Semantic Web\\[6pt] Cooperative
  Action in Digital Change \vskip1em}

\subtitle{Lecture in the Module 10-202-2309\\ for Master Computer Science}

\author{Prof. Dr. Hans-Gert Gräbe\\
\url{http://www.informatik.uni-leipzig.de/~graebe}}

\date{February 2022}
\begin{document}

{\setbeamertemplate{footline}{}
\begin{frame}
  \titlepage
\end{frame}}

\begin{frame}{Network Cooperation. The History of the .NET Project}

  \ueberschrift{Microsoft on the Way to Modes\\ of Network Cooperation -- the
    .NET Project}

  \textbf{What is .NET}
  
"... completely redefining the way Microsoft will do business in the future
  ... and how software should be developed." (Westphal, 2002)
\begin{itemize}
\item The platform should replace the previous art of Windows programming,
  flexibly access operating system and basic functions and support exchanges
  between programs.
\item Designed for use on different hardware platforms down to cell phones and
  PDAs. The Java idea without restriction to Java as programming language.
\end{itemize}
\end{frame}

\begin{frame}{The History of the .NET Project}
\textbf{Prehistory:}

Legal dispute between Sun and Microsoft over Java
\begin{itemize}
\item Microsoft is expanding Java according to its own ideas and needs and
  thus endangers Java compatibility
\item Microsoft implementations J++ and J\#
\end{itemize}

Further problems:
\begin{itemize}
\item Also those languages as Visual Basic, C++, and J++ mostly used for
  Windows programming languages were not binary compatible.
\item Even string data types weren't binary compatible -- .NET is consistently
  Unicode based.
\item No uniform model of memory management.
\end{itemize}
\end{frame}

\begin{frame}{The History of the .NET Project}
\begin{itemize}
\item 1996: first work on .NET
\item 2000: .NET Framework 1.0 Beta
\item October 2000 -- C\# and the CLI are submitted by MS, HP and Intel for
  standardisation to the ECMA
  \begin{itemize}
  \item ECMA -- European Computer Manufacturers Association
  \item December 2001 -- First standard passed to ISO
  \item April 2003 -- Adoption of the ISO standards ISO/IEC 23270 (C\#) and
    ISO/IEC 23271 (CLI).
  \end{itemize}
\item April 2003 -- Delivery of .NET Framework 1.1 together with Windows
  Server 2003, which provides an integrated .NET runtime environment.
  \begin{itemize}
  \item Thus transition to the new platform at the conceptual level of
    Corporate Servers. However, integration into the whole product family is
    not advancing as quickly as expected.
  \end{itemize}
\item End of 2006: .NET 3.0, later an integral part of Windows Vista and
  Windows Server 2008, with profound, also conceptual extensions of the
  architecture.
\end{itemize}
\end{frame}

\begin{frame}{The History of the .NET Project}
\begin{itemize}
\item End of 2007: Visual Studio 2008 and .NET Framework 3.5
  \begin{itemize}
  \item Framework Class Library (FCL) -- comparable to the Java Base Classes
    that are shipped with any Java distribution -- includes almost 12\,000
    classes in 300 namespaces.
  \item Partial release of the source code of the Base Class Library under the
    restrictive Microsoft Reference Source License.
  \end{itemize}
\item April 2014: Microsoft announces the creation of an independent .NET
  Foundation at -- \url{http://www.dotnetfoundation.org}
  \begin{itemize}
  \item January 2015: Announcement of the .NET Open Source Initiative.
  \item Stronger separation between .NET Framework and .NET Core. .NET Core
    contains the base classes and the runtime environment. Their further
    development will be transferred to the .NET Foundation.
  \end{itemize}
\end{itemize}
\end{frame}

\begin{frame}{.NET Project and Open Source}
ECMA standardization also allows implementation of the standard
on other platforms.\vspace{-1em}

Versions beyond Windows:\vspace{-1em}
\begin{itemize}
\item Microsoft itself with the Shared Source CLI released in 2002 versions
  for Mac OS and FreeBSD. These activities were later abandoned.
\item Various activities of the Linux community to implement the concepts and
  create a free .NET version.
  \begin{itemize}
  \item In 2009 the dotGNU project starts to implement a runtime
    environment for Portable.NET. Developed upto a release version 0.1 and
    discontinued at the end of 2012: "As of December 2012, the DotGNU project
    has been decommissioned, until and unless a substantial new volunteer
    effort arises.“
  \end{itemize}
\item Much behind the capabilities of the Windows versions.
\item The only powerful "free" project is the Mono Project
  \url{http://www.mono-project.com/}
\end{itemize}
\end{frame}

\begin{frame}{The History of the Mono Project}
\begin{itemize}
\item In 1999, Miguel de Icaza and Nat Friedman founded the company
  \emph{Helix Code}.  The company was renamed to \emph{Ximian} in 2001.
\item Business model: Solutions and services, based on a mix of free and
  commercial software.
\item Involved in the creation of the Linux Gnome project.
\item 2002 start of the Mono project.
\item Company was acquired by \emph{Novell} in 2003, which continues to
  strenghten its Linux portfolio.
\item In 2011, Novell was acquired by the \emph{Attachmate Group} who has no
  interest in the continuation of the Mono project.
  \begin{itemize}
  \item After several months of discussions, the US Department of Justice
    (DOJ) and the German Federal Competition Office (FCO) have allowed a
    consortium of Microsoft, Oracle, Apple and EMC to acquire 882 patents from
    Novell only subject to conditions clearly intended to prevent their use
    against free software players.  (FSFE Newsletter, April 2011)
  \end{itemize}
\end{itemize}
\end{frame}

\begin{frame}{The History of the Mono Project}
\begin{itemize}
\item In 2011, Icaza and Friedman founded \emph{Xamarin}
  \url{http://xamarin.com} and there bundle the further development on the
  Mono Project.
  \begin{itemize}
  \item The company's focus is on mobile applications.
  \end{itemize}
\item The mono core, the runtime environment, is freely available under the
  LGPL v.2, but Xamarin also offers commercial licenses for the Mono
  platform. 
  \begin{itemize}
  \item If you are planning to use Mono as a bundled part of your commercial
    product, on embedded hardware, or in any other situation where using the
    LGPL-licensed Mono is impossible or problematic, Xamarin can sell you a
    commercially-friendly license that will suit your needs.
  \item Many commercial users of Mono acquire a commercial license when they
    want the flexibility and peace of mind to use Mono without worrying about
    the terms of the LGPL.
  \end{itemize}
\item New stage of cooperation: at the end of 2013, Microsoft, Xamarin and
  others create the \emph{.NET Foundation}.
\end{itemize}
\end{frame}

\begin{frame}{.NET Open Sourcing}
\begin{itemize}
\item In 2008 Microsoft published the source code of the framework under the
  restrictive Microsoft Reference License.
\item At the end of 2013, Microsoft, Xamarin and others founded the \emph{.NET
  Foundation} as the new rights holder and licensor of .NET Frameworks.
  \url{http://www.dotnetfoundation.org/}
  \begin{itemize}
  \item In 2007 Microsoft still claimed that the Mono project violated
    Microsoft's IP rights.
  \end{itemize}
\item At the end of 2014, a subset of the Reference Source source code is made
  available on GitHub and published under the MIT license.
  \begin{itemize}
  \item \url{https://github.com/dotnet}
  \item This was done to fill gaps between Mono and .NET using the same code.
  \end{itemize}
\end{itemize}
\end{frame}

\begin{frame}{.NET Open Sourcing}
\begin{itemize}
\item At the same time, Microsoft has started also to publish the revised
  components of the framework under the name \emph{.NET Core} on GitHub under
  the MIT license.
\item Basis for the upcoming, modular \emph{.NET Framework 5}.
  \begin{itemize}
  \item .NET Core has been transferred from Microsoft to the .NET Foundation
    been.
  \end{itemize}
\item Using the MIT license, there are in fact no more restrictions how to use
  the source code of .NET Core.
  \begin{itemize}
  \item With the establishment of the .NET Foundation and the transfer of
    rights and source codes to the Foundation, Microsoft works actively with
    Xamarin, to provide .NET on different platforms. By disclosing the source
    code under the MIT license or Apache 2.0 license the source code of the
    .NET Framework can be used almost arbitrarily -- even in closed source
    projects. Licensing and patent law disputes are therefore hardly possible
    any more and no longer to be feared. (Wikipedia)
  \end{itemize}
\end{itemize}
\end{frame}

\begin{frame}{Conditions of Cooperative Action}
  
Which \emph{legal requirements} for the civil society are constitutive for
cooperative contexts?\vspace{-1em}
\begin{itemize}
\item \emph{Freedom of contract} as the right to establish contexts of
  cooperative agreements.
\item The \emph{right to free speech} (as an internal right) precedes the
  freedom of contract.
  \begin{itemize}
  \item This right has nothing directly to do with the concept of democracy.
  \end{itemize}
\item Both presuppose the (mental and social) \emph{ability to close
  contracts} and thus a society of owners. (Legal capacity --
  Geschäftsfähigkeit) 
\item \emph{Prohibited direct intervention from outside} on the inside of
  cooperative contexts as \emph{social normative}.
  \begin{itemize}
  \item Such a right on the \emph{private level} is part of the personality
    rights (Persönlichkeitsrecht -- right to privacy as a personality right in
    the Constitution) and a cultural achievement of the civil society.
  \end{itemize}
\end{itemize}
\end{frame}

\begin{frame}{Conditions of Cooperative Action}
  
Results of dynamics in the internal relationship are as topoi visible from the
outside.
\begin{itemize}
\item Example of corporate identity.
\item Consequence of the prohibition of intervention.
\end{itemize}
Inside is outside in relation to almost everyone else.
\begin{itemize}
\item Foreign topoi appear as conditions of action, whose \emph{dynamics} are
  only accessible to the extent that this process can be internalized via a
  translation (justified expectations).
\end{itemize}
\end{frame}

\begin{frame}{Cooperation and Competition}
Cooperation and competition (Kooperation und Konkurrenz) are available as
forms of structuring of society on the same logical level.
\begin{itemize}
\item Only \emph{parts} of bundle of interests are used in cooperative ties,
  other interests remain \emph{competitive} (concurrent).
\item \emph{Concurrent} means more concurrency than opposition, clash.
\item System theory: positive and negative feedback.
\item Debate about (German) Kooperenz,
  \url{http://www.frei-gesellschaft.de/wiki/Kooperenz}
\end{itemize}
„The area of tension between cooperation and competition is the tension
between the possibility of cooperation and the possibility of demarcation and
thus the field of tension between two pillars of the civil legal system --
Freedom and Property“. (E. Moglen)
\end{frame}

\begin{frame}{Cooperation and Competition}
Cooperation and competition thus appear as two poles a continuum of possible
forms.\vspace{-1em}
\begin{itemize}
\item \textbf{Cooperation:} narrow interests, high depth of justification,
  coupling already in the \emph{planning phase} of the action.
\item \textbf{Competition:} broad range of interests, low depth of
  justification, coupling only occurs in the course of \emph{action
    execution}.
\end{itemize}\vspace{-1em}
The balance of the weights between the two poles are constantly changing.
Regional regulatory and legal areas (e.g. states) are competing social
practices where these weights are differently balanced.\vspace{-1em}

\begin{itemize}
\item (inner) bourgeois "cultures".
\end{itemize}\vspace{-1em}
In this understanding, \textbf{Open Culture} is a \emph{specific bourgeois
  cultural practice} in which cooperative moments are valued higher than in
currently common (e.g. neoliberal) practices.
\end{frame}

\end{document}
