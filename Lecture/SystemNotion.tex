\documentclass{beamer}
\usepackage{lsfolien}
\usepackage[english]{babel}
\usepackage[utf8]{inputenc}

\myfootline{System Modelling and Semantic Web -- Summer Term 2022}{Hans-Gert
  Gräbe} 

\newcommand{\ueberschrift}[1]{\begin{center}\bf #1\end{center}}

\title{Modelling Sustainable Systems\\ and Semantic Web\\[6pt]
  Systems and their Development
  \vskip1em}

\subtitle{Lecture in the Module 10-202-2312\\ for Master Computer Science}

\author{Prof. Dr. Hans-Gert Gräbe\\
\url{http://www.informatik.uni-leipzig.de/~graebe}}

\date{April 2022}
\begin{document}

{\setbeamertemplate{footline}{}
\begin{frame}
  \titlepage
\end{frame}}

\begin{frame}{What is a System?}
See also the Handout on "Systems, Organisations, Mangement".

\begin{block}{System definition}
  A \emph{system} is a \emph{delimited set of components}.  Their interaction
  realises a \emph{qualitatively new function} (emergent property) and thus
  constitutes \emph{a new unified whole}.
\end{block}

\emph{Example:} An aeroplane consists of many parts. None of them can fly,
only the aircraft as a unified whole can fly. Flying is an emergent property
of the system as a whole only.
  
The systemic approach is a form of reduction of the complexity of the totally
interconnected real-world (\emph{reality} for short).
\end{frame}

\begin{frame}{What is a System?}
  \begin{block}{Three dimensions of reduction to the essentials.}
    \begin{itemize}
    \item[(1)] Outer demarcation of the system against an \emph{environment}
      (the \emph{context}), reduction of these relationships to input/output
      relationships (specifications and interfaces) and guaranteed throughput.
    \item[(2)] Inner demarcation of the system by combining subareas to
      \emph{components}, whose functioning is reduced to “behavioural control”
      via input/output relations (specifications and interfaces).
    \item[(3)] Reduction of the relations in the system itself to “causally
      essential” relationships.
    \end{itemize}
  \end{block}
\end{frame}

\begin{frame}{What is a System?}
This reduction is essential for both the \emph{description} of real-world
contexts and for its \emph{operational control}.

In this understanding, systems are to be considered both as a \textbf{unit of
  analysis/description (modelling)} and \textbf{a unit of execution
  (operation)}.

Every system approach contains a \textbf{central contradiction:} Analysis
emphasises the \emph{structuredness} and thus \emph{decomposability} of the
system. Operation emphasises the \emph{interdependence} and thus
\emph{indecomposability} of the system.

In the assembled system in addition to the components, the \emph{connections}
also play an important role.  They mediate the \emph{flow of energy, material
  and information} which is required for the operation of each component and
thus for the viability of the system.
\end{frame}

\begin{frame}{Systems and Practice}
Systems are a part of the complex relationships of practice.

Analysis and modelling as human thinking activities do not only have a
\emph{mental quality}, but also a \emph{practical} one. The goal of modelling
is to influence real-world processes.

In such an understanding, the \emph{criterion of "truth"} is not related to an
"idea" that precedes all experience, but rooted in the practical experience
itself.  It expresses, how successful the \emph{justified expectations} from
the mental modelling and the \emph{experienced results} of the transformation
of the real-world according to that plan are related.
\end{frame}

\begin{frame}{Practice and Language}

However, this experience itself can only be communicated intersubjectively in
language form and has thus itself mental quality.

A coherent concept of a system must combine both dimensions -- language shapes
expressiveness of practical cooperative action, expressive practices shape
language.

Thus language means above all the \emph{common conceptual penetration} of
real-world processes in cooperative action and thus the \emph{formation of
  common conceptual worlds} in cooperative action spaces.

\end{frame}

\begin{frame}{Systems and Autonomy}

The linguistic concentration on essentials in the description suggests a
certain closedness (\emph{autonomy}) of a system.
  
However, this autonomy is always \emph{relative} to existing or assumed
\emph{operating conditions} provided by the environment, the \emph{context}.

System descriptions are therefore never absolute, but \emph{always to be
  contextualised}, even if this contextualisation itself often falls victim to
the reduction to essentials.

\end{frame}

\begin{frame}{Components are for Composition}

In this understanding, the environment of a system is itself systemically
structured, even if it may still be insufficiently mentally conceptionalised.
This raises the question of the \emph{genesis and development of systemic
  descriptive structures}.

The system concept is well suited for such a task since it is self-similar --
components can again be considered and analysed as systems.  However, this is
done under a \emph{different reduction to essentials} than for the system
composed from these components.

In such an understanding, the \emph{main task of systemic modelling} is the
composition of components into new systems (system genesis) or the improvement
of the interaction of components in a system (system development).

\end{frame}

\begin{frame}{Viable Systems are Open Systems}
\vskip1em
The feedback loop of justified expectations and experienced results which
drives systemic development presupposes \emph{"living" systems}, i.e.  systems
which are \emph{implemented} and \emph{operated} in the real world.

For the operation of a system a qualitatively and quantitatively determined
\textbf{throughput of energy, matter and information} is required.

On the other hand, such a throughput has great influence on the inner
structure of a system.

\emph{Examples:}\vspace*{-1em}
\begin{itemize}
\item Rayleigh-Bénard cells  \url{https://www.youtube.com/watch?v=9Ru_bjW3eL8}
\item Metabolism of biological systems
\item The "global" geo-biological system Earth 
\end{itemize}
\vskip1em  
\end{frame}

\begin{frame}{Development Patterns of (Technical) Systems}

\begin{block}{Altshuller's Law of Energy Conductivity.}
  A necessary condition for the viability of a technical system is the flow of
  energy through all its parts.
\end{block}

This external throughput is often constitutive for the inner system structure
through resonance and dissonance phenomena of amplification and extinction of
inner dynamics.

\begin{block}{Altshuller's Law of Coordination of the Rhythm of the Parts.}
  A necessary condition for the viability of a technical system is the
  coordination of the rhythms of all parts of the system.
\end{block}
\end{frame}

\begin{frame}{Changing the World in a Planned Way}
The system concept is a tool in the hands of humans who want to \textbf{change
  the world} following their (intentions, goals and) plans.

\begin{block}{11th Feuerbach Thesis (Marx)}
  Philosophers have only interpreted the world differently, it is a matter of
  changing it.\\[4pt] Die Philosophen haben die Welt nur verschieden
  interpretiert, es kömmt darauf an, sie zu verändern.
\end{block}

But: The world is in constant motion and also permanently changes itself.

So more precisely: It is a matter of \textbf{influencing the development of
  reality}.
\end{frame}

\begin{frame}{System Development}

Two practical tasks:
\begin{itemize}
\item[(1)] Build new system
\item[(2)] Improve existing system
\end{itemize}

(1) can be consideres as a special case of (2), since every need for a new
system comes with at least rough ideas about that new system, so there is also
under (1) an at least rough description form of the system to be created.

\begin{center}
  \includegraphics[width=.9\textwidth]{images/SD-1.png}
\end{center}
\end{frame}

\begin{frame}{System Development}
  This basic scheme fits not only technical systems, but also the modelling of
  social, socio-ecological and cultural systems, so it is sufficiently
  universal.

How does such a system evolve over time?
\begin{center}
  \includegraphics[width=.9\textwidth]{images/SD-2.png}
\end{center}
\end{frame}

\begin{frame}{System Development}
\begin{center}
  \includegraphics[width=.9\textwidth]{images/SD-3.png}
\end{center}
Transitional development as \emph{different versions} of the system over the
time.
\end{frame}

\begin{frame}{System Development}
  \begin{minipage}{.45\textwidth}
    But this can also be understood as development in time \emph{of the same
      system}.\vskip1em

    Transitional management versus adaptive management.
  \end{minipage}\hfill
  \begin{minipage}{.45\textwidth}
    \begin{center}
      \includegraphics[width=.6\textwidth]{images/SD-4.png}
    \end{center}
  \end{minipage}
\end{frame}

\begin{frame}{Development of Systems}
  The development of a system can therefore be conceived as a contradiction
  between an \emph{ideal line of development} and a \emph{real line of
    development}.

This idea is reflected in the \textbf{TRIZ concept of the \emph{Ideal Final
    Result}} (IFR -- Ideales Endresultat).

In the Theory of Dynamical Systems, system development is conceived as a
progression of states, which can be described by a function $f(t)$ with values
in a phase space.

The \emph{ideal behaviour} is described by mathematical relationships, such as
differential equations of the laws of motion and geometrically displayed as
\emph{trajectory}. 

\end{frame}

\begin{frame}{Trajectories and Limit Cycles}

  \begin{center}
    \includegraphics[width=.8\textwidth]{images/LimitCycles.png}

    \footnotesize
    Source: \url{https://de.wikipedia.org/wiki/Datei:VanDerPolPhaseSpace.png}
  \end{center}
  
\end{frame}

\begin{frame}{System Dynamics}
These differential equations and trajectories are part of the
\emph{description form of the system} and thus have already been created by
\emph{reduction to essentials}.

In the modelling it is assumed that everything essential is taken into
account, i.e. that the \emph{real temporal development} $r(t)$ of the system
(as \emph{experienced result}) differs from the \emph{ideal temporal
  development} $f(t)$ (as \emph{justified expectation}) only by a small
difference $d(t)=r(t)-f(t)$, which is \emph{insignificant for the selected
  essential}.

While $f(t)$ enables a \emph{quantitative prediction} of the development of
the system, the statement that $d(t)$ is "small" or "damped" is a
\emph{qualitative statement} of the description form itself.

\end{frame}

\begin{frame}{System Dynamics}
Often one also restricts oneself with $f(t)$ to a \emph{qualitative statement}
about the exact position of the trajectories as invariants in the solution
space and thus to the statement that $r(t)$ oscillates around these
trajectories in a damped manner.

These trajectories seem to "magically" attract the real states and are
therefore also called \emph{attractors} (steady state equilibrium).

For example, the Earth moves on an elliptical orbit around the Sun in the
sense that real deviations from this orbit are compensated.  Such patterns
often only become visible when systemic motion is approximately repeated in
such \emph{steady states}.
\end{frame}

\begin{frame}{System Dynamics}

How do small disturbances influence the path behaviour in a quasi-periodic
motion? An important answer is given by the Kolmogorov-Arnold-Moser (KAM)
theorem.

From the English Wikipedia: \footnotesize

The KAM theorem states that if the system is subjected to a weak nonlinear
perturbation, some of the invariant tori are deformed and survive ..., while
others are destroyed (by even arbitrarily small perturbations ...).  Surviving
tori meet the non-resonance condition, i.e., they have “sufficiently
irrational” frequencies. This implies that the motion ... continues to be
quasiperiodic, with the independent periods changed (as a consequence of the
non-degeneracy condition). The KAM theorem quantifies the level of
perturbation that can be applied for this to be true.\\[.3em]
KAM tori that are destroyed by perturbation become invariant \emph{Cantor
  sets} ...\\[.3em]
The non-resonance and non-degeneracy conditions of the KAM theorem become
increasingly difficult to satisfy for systems with more degrees of freedom.

\end{frame}

\begin{frame}{Coevolution of Systems}
\begin{minipage}{.55\textwidth}
  What is the relationship between the development
  \begin{itemize}
  \item of the system itself,
  \item of the components in the system and
  \item of the relationships in the system?
  \end{itemize}
\end{minipage}\hfill 
\begin{minipage}{.4\textwidth}
    \begin{center}
      \includegraphics[width=.95\textwidth]{images/SD-5.png}
    \end{center}
  \end{minipage}
\end{frame}

\begin{frame}{Coevolution of Systems}

The coupling of developments between components is driven by resonance
phenomena and coupled to eigentimes and eigenspaces of the inner equations of
motion, different for different components.\vskip2em

\begin{block}{Altshuller's Law of Non-uniformity of the Development of the
    Parts of a System.}  The parts of a system develop non-uniformly; the more
  complicated the system, the more non-uniformly develop its parts.
\end{block}
\end{frame}

\end{document}
