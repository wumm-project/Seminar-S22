\documentclass{beamer}
\usepackage{lsfolien}
\usepackage[english]{babel}
\usepackage[utf8]{inputenc}

\myfootline{System Modelling and Semantic Web -- Summer term 2022}{Hans-Gert
  Gräbe}

\newcommand{\ueberschrift}[1]{\begin{center}\bf #1\end{center}}

\title{Modelling Sustainable Systems\\ and Semantic Web\\[6pt]
  Digital Action Space  \vskip1em}

\subtitle{Lecture in the Module 10-202-2312\\ for Master Computer Science}

\author{Prof. Dr. Hans-Gert Gräbe\\
\url{http://www.informatik.uni-leipzig.de/~graebe}}

\date{May 2022}
\begin{document}

{\setbeamertemplate{footline}{}
\begin{frame}
  \titlepage
\end{frame}}


\begin{frame}{Summary}

We introduced so far several central notions.\vskip2em
  
\textbf{System} as an essential structural form of delimitation and
containment both in reality (from biological cells to social forms such as
companies as economically delimited action spaces or states as politically and
legally delimited action spaces -- interestingly, this does not continue to
the quantum level) and structures of delimitation and reduction of complexity
in cooperative thinking actions.
\end{frame}
\begin{frame}{Summary}

\textbf{Systemic structures} as a set of interdependent open systems whose
throughputs of energy, substance and information are mutually dependent,
without the interrelationships already being understood as an overall system.

\emph{Special case:} Short scale internal rhythms of the systems, long scale
rhythms of the throughput relationships. Then the dynamics of evolution at
both system levels can often still be analysed with advanced mathematical
methods (interlacing of micro- and macroevolution), whereby new cooperative
phenomena emerge.

For more details on systemic structures:
\begin{itemize}
\item Russell L. Ackoff (1971). Towards a system of systems concept.
  Management Science 17 (11), 661-671.
\end{itemize}
\end{frame}
\begin{frame}{Summary}

Shchedrovitsky describes the genesis of such an overall system as a
\textbf{schematisation} as follows:
\begin{itemize}
\item \textbf{Analysis:} Structural decomposition of the overall system into
  its parts, (recursive) schematisation of the parts.
\item \textbf{Synthesis:} Combining the descriptions of the parts and their
  interactions, reduced to essentials, into a description of the overall
  system that is oriented towards a \textbf{specific objective} and thus a
  \textbf{specific contextualisation}.
\item \textbf{Evaluation:} Does this description prove its value in the
  feedback cycle with the practice?

  Iterative continuation on the basis of the modification of the objective of
  the mental process.
\end{itemize}
In view of the specificity of software as a thought artefact, such approaches
also play a central role in Software Engineering in Component Software
(module, integration and system level).\vskip1em

\end{frame}
\begin{frame}{Summary}

In this process, both structural and operational approaches are important.
\begin{itemize}
\item \textbf{Analysis} focuses on delimitation and decomposition.  Structural
  approaches and a spatial metaphor are in the foreground.
\item However, the system is only operable in assembled state.
  \textbf{Synthesis} is directed towards relational structures. Operational
  approaches as complexity reduction and relational metaphors are in the
  foreground.
\end{itemize}

\textbf{Action Spaces:} This term breaks down these systemic structuring
processes to social systems with specific attention to the actors involved as
\textbf{subjects}.
\end{frame}

\begin{frame}{What is Data?}
The last question in the last lecture was about data. 
  
  \begin{itemize}
  \item Data as a specific form of description.
  \item Capturing data always means choosing what \emph{not} to capture.
  \item Hence capturing data is subjective \emph{from the position of a
    subject}.
  \item Data as a link between world and reality.
  \item But what then is \emph{objective} data?
    \begin{itemize}
    \item Specific reflex of a positivistic understanding of science.
    \item Use and misuse: Such an understanding (of science) is an important
      cultural achievement of humankind, which, however, also has to be
      \emph{contextualised in concrete-historical terms}.
    \end{itemize}
  \item Data is also a form of cooperative practices of people as subjects.
  \end{itemize}
\end{frame}

\begin{frame}{Digital Transformation}
  Concept of the \textbf{Digital Universe} as a rather technically shaped
  inner-societal space of action through the processing of digital data, with
  a vague demarcation.  Picking up a common buzz word.
\begin{itemize}
\item "By 2020, the digital universe will amount to 44 trillion gigabytes"
  (EMC Digital Universe with Research \& Analysis by IDC. The Digital Universe
  of Opportunities: Rich Data and the Increasing Value of the Internet of
  Things. April 2014).
\item Reference to the central thesis -- a spatial metaphor is used to analyse
  the digital transformation from a specific dichotomy.
\end{itemize}
\begin{block}{Central Thesis:}
  The digital transformation is characterised by a rapidly growing "world of
  digital data", through the analysis and processing of which influence is
  exerted on real-world processes.
\end{block}
\end{frame}
\begin{frame}{Digital Transformation}
  \ueberschrift{On the Critique of this Approach}
  \begin{itemize}
  \item In this version, we want to focus on questions of how current
    structuring processes in the digital universe and real-world processes
    interact and influence each other.
  \item The concept of juxtaposing "real-world" and "digital" reality is
    problematic overall, since actions in the digital universe are both
    motivated by real-world practices and have an influence on real-world
    practices.
 \item However, the concept emphasises that many real-world contexts of action
   interact with technical processes in this space and therefore such an
   abstraction seems reasonable.
  \end{itemize}
\end{frame}
\begin{frame}{Digital Transformation}
  \ueberschrift{The Digital Knowledge Revolution}

Michael Schetsche: "The digital knowledge revolution" (2006, in German)
identifies six social and cultural dimensions:
\begin{itemize}
\item a new order of knowledge,
\item social control through technical norms,
\item the automatic archive function of the net,
\item the supplementation of the exchange economy by a gift economy,
\item the abolition of the guiding difference between "public" and "private“, 
\item the dialectic of possibility and obligation of permanent communication.
\end{itemize}
\end{frame}
\begin{frame}{Digital Transformation}

All in all, it makes sense and is necessary to speak of a \emph{transformed
  social order} in which the \emph{structurally decisive changes} emanate from
the digital networks.

  A more precise understanding of the change in particular in the order of
  knowledge is an essential part of an analysis of the digital transformation.

  Problem: For the new phenomena, we (initially) only have the old terms.

  I will not elaborate on that here and refer to (Schetsche 2006).

\end{frame}

\section{Digital Action Spaces}
\begin{frame}{Digital Action Spaces}\centering\Large\bf

  How and where are you acting\\ in the digital universe?

  What opportunities for your own\\ and collective action in the digital
  universe\\ do you frequently use?

  Which preconditions\\ must be fulfilled for this?
\end{frame}
\begin{frame}{Digital Action Spaces. From earlier Discussions}
  \begin{itemize}
  \item The digital universe breaks down into different universes -- the
    Instagram universe, the Facebook universe, the Google Scholar universe,
    the Wikipedia universe, the Search universe etc.
    \begin{itemize}
    \item Space in space metapher. Such „subspaces“ are constituted by
      specific kinds of social relations and specific social practices.
    \end{itemize}
  \item What to do there?
    \begin{itemize}
    \item Upload pictures and data.
    \item Like and be liked.
    \item Communcate with friends in Corona times.
    \item Online appointment for offline meeting.
    \item Present oneself in digital spaces.
    \item Searching for useful information.
    \end{itemize}
  \end{itemize}
\end{frame}
\begin{frame}{Digital Action Spaces. Accounts}
  \begin{itemize}
  \item Diversity of accounts = diversity of digital identities
    \begin{itemize}
    \item Identity in the singular or in the plural?
    \item My Core -- world and reality, meaningful terms?
    \item Diversity of identities or of real-world facets
    \end{itemize}
  \item \emph{Identity} as an important concept in the civil legal system,
    which is also legally attached in order to be able to assign consequences
    of actions.
\item Questions of private digital spaces of action can only be meaningfully
  discussed if the user is "logged in" to a computer via an \textbf{account}.

  This also applies to other (e.g. mobile) devices, although the technical
  connection to an account (via SIM card and own security settings) is less
  visible there.
  \end{itemize}
\end{frame}
\begin{frame}{Using Digital Action Spaces. Digital Identity}  
\begin{itemize}
\item Such an account is associated with a \textbf{digital identity} to which
  actions on the internet are assigned, via which the usual legal-social
  constructs of the \emph{legal attributability of actions} are transferred to
  the digital sphere.  
  \begin{itemize}
  \item The private attribution of consequences of action is a \emph{pillar of
    the civil legal order}.
  \item The technical possibilities in the digital universe can \emph{improve}
    or \emph{complicate} the attributability of legal responsibility.
  \item Possibility of \emph{anonymous action}. But: traces of actions are
    fundamentally accessible to forensic analysis. This also applies to
    actions on the internet.
  \end{itemize}
\end{itemize}
\end{frame}
\begin{frame}{Real-world and Digital Identities}
  \begin{center}
    \includegraphics[width=.9\textwidth]{images/DI-4.png}
  \end{center}
  For actions in the digital universe, real-world identities must be tied to
  digital identities.
\end{frame}
\begin{frame}{Real-world and Digital Identities}
\begin{itemize}
\item The assignment of a digital identity to a real person takes place via
  \textbf{authentication}, which appears to be a \emph{private} act (albeit
  technically preconditioned).
  \begin{itemize}
  \item However, it presupposes an \textbf{authenticator} as the technical
    counterpart and thus a higher-level legal context. This assignment process
    is nevertheless postulated as private in the public.
  \end{itemize}
\item Private digital spaces of action can only be shaped through the
  binding to a digital identity.
  \begin{itemize}
  \item The rebinding of a digital identity to a civic legal subject is itself
    a socio-technically institutionalised process.
  \item This rebinding is particularly simple if the signature of a technical
    artefact from the digital universe can be easily assigned to the civil
    legal subject.
  \end{itemize}
\end{itemize}
\end{frame}
\begin{frame}{Acting on the Internet}\small
\begin{itemize}
\item Action spaces are socially determined. Digital action spaces can be and
  are constituted and assigned through \textbf{authorisation}.
\item In shaping action spaces on the internet, subjects are highly dependent
  on technical services and thus on external institutions whose
  \emph{trustworthiness} they must assess appropriately.
\item Regulatory provisions for actions on the internet exist only in
  rudimentary form, so that \emph{appropriate practical action} and
  \emph{cooperative arrangements} on a \emph{contractual basis} are the main
  forms of shaping a concept of "privacy on the internet".
\item An \emph{appropriate} understanding of the technical conditions,
  possibilities and restrictions of the internet is essential for the
  qualified shaping of personal actions on the internet.
\item Social action constitutes the intersubjective relations of a subject.
\end{itemize}
\end{frame}
\begin{frame}{On the Concept of Action Space}
  \begin{block}{Thesis:}
    The concept of action space in the nowadays common sense is a cultural
    achievement of bourgeois civic society.
  \end{block}\small
\begin{itemize}
\item Action spaces as a "space within space" structure and thus contextualise
  possibilities of cooperative arrangements in an "external space".
\item \emph{My} action spaces are identity-constituting, and the actions in
  these spaces form the basis for my personality as a civic legal subject.
\item Only on this basis delimitations of other concepts such as
  \emph{environment}, \emph{acting in an environment}, \emph{cooperative
    action} and thus ultimately concepts such as \emph{subject},
  \emph{privacy} and \emph{identity} can be meaningfully grasped.
\item Collaborative action spaces can be condensed into "cooperative subjects"
  in the sense of the civil legal order.
\end{itemize}
\end{frame}
\begin{frame}{Private Action and (Digital) Identity}

\emph{Private action} presupposes a \emph{concept of self}, of personal
\emph{identity}.
\begin{itemize}
\item Digital identity, multiple digital identity and roles
  \begin{itemize}
  \item[] Is identity divisible?
  \end{itemize}
\item Abstract identity, textual representation
  \begin{itemize}
  \item[] Assignment mechanisms, e.g. website and login
  \end{itemize}
\item Authentication
  \begin{itemize}
  \item[] Password, other forms of authentication
  \end{itemize}
\item Authorisation
  \begin{itemize}
  \item[] Me as subject and as object of authorisation.
  \end{itemize}
\item Potential and real assignment. Notion of session.
\end{itemize}
\end{frame}
\begin{frame}{Digital Identities}
\begin{itemize}
\item Digital identity, abstract identity, textual representation
\item Website, login, mobile devices
\item Concept of session (not only on websites)
\item Authentication and authorisation
\end{itemize}
\begin{block}{Digital Identity}
  In the following, we understand \emph{Digital Identity} as a
  \textbf{real-world civic subject} which is \emph{authenticated} under a
  textual representation \texttt{<name@computername>} and \emph{authorised} in
  the context of a session, that performs actions in the digital universe for
  a limited period of time.
\end{block}
\end{frame}
\begin{frame}{Digital Identities and Roles}
  \ueberschrift{The Concept of Roles in Computer Science}
  
\begin{itemize}
\item In computer science, a role is a bundle of necessary \emph{experience,
  knowledge and skills} that an employee must have in order to perform a
  certain \emph{activity}.
\item Roles are defined by \emph{role descriptions} within a \emph{role
  model}.
\item A role is associated with \emph{activities} and \emph{responsibilities}.
\item \emph{Qualification characteristics} are required to perform a role.
\item A person can have several roles. Several persons can have the same role.
\end{itemize}
\end{frame}
\end{document}
