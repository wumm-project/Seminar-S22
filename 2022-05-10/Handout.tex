\documentclass[11pt,a4paper]{article}
\usepackage{ls}
\usepackage[english]{babel}
\setlist{noitemsep}

\title{P-TRIZ in the History of Business Processes}  

\author{Hans-Gert Gr\"abe}
\date{May 08, 2022}

\begin{document}
\maketitle

\section{Development of Business Process Modelling (BPM)}

Smith, Part 3:

The history of BPM is long and rich. It began in the 1920s and was dominated
by Frederick Taylor’s "The Principles Science of Management" ...

In a second wave, industrial processes were \textbf{manually reengineered}
and, through a one-time activity, cast in concrete in the bowels of today’s
packaged enterprise applications technology.

In a third wave of BPM, \textbf{executable digitized processes} are now freed
from their castings as engrained software to re-emerge as a flexible new form
of process data.

An era of \textbf{process manufacturing} has been ushered in. With these new
capabilities at hand, attention is turning once again towards process
innovation.

Each era of BPM has added new capabilities to the last. For example, BPM
systems enable process architects to readily deploy creative new process
designs, sidestepping time and resource intensive implementation projects of
the past that so denuded and distorted reengineering of its creative
potential. Now, P-TRIZ is an emerging method that builds on the shoulders of
those giants.

P-TRIZ is the application of modern TRIZ towards business process improvement,
innovation, and transformation. Coupled to BPM methods, it provides the
engineering discipline that amplifies the creativity of those who seek to
re-design processes.

\subsection{1990: Business Process Reengineering (BPR)}
Creativity is directed on process articulation. 
\begin{itemize}
\item Process lateral thinking
\item Reengineering techniques
\item Decades old technique for re-inventing business processes
\item Based on management wisdom, creativity, common sense change management
  and rules of thumb
\end{itemize}
Development of process landscapes, reference models for processes such as the
ISO 9000 model
\begin{center}
  \includegraphics[width=.8\textwidth]{images/PM-ISO9000.png}
\end{center}

\subsection{2000: Business Process Management Systems (BPMS)}
Creativity is directed on process technology. 
\begin{itemize}
\item Executable processes
\item Direct path to implementation
\item Management of the process lifecycle
\item Process modelling languages
\item The discipline of using process models as a direct basis for the
  implementation of IT systems in support of those processes.
\item The use of IT solutions (BPMS) to govern the lifecycle of process
  improvement: discovery, design, deployment, execution, operation, change and
  optimisation.
\item A digital representation for business process (“process as data”)
  opening the door to manipulation, generation, transformation, not just
  automation.
\item Obliterating a business-IT divide – empowering the business user. 
\end{itemize}

\subsection{2005: Business Process Innovation (P-TRIZ)}
Creativity directed on process methodology
\begin{itemize}
\item Exhaustive generation of re-design options
\item Systematically solve problems in processes
\item Use of modern innovation methods (P-TRIZ) to accelerate and systematise
  the generation of reengineering options.
\item Amplifies the creativity of practitioners to exhaustively explore
  re-design alternatives
\item Knowledge management (problem-solution) of process patterns
\item A reliable and guided process to resolve contradictions in process
  design
\end{itemize}

\subsection{200?: Business Process Realisation (Transformation)}
Very hypothetical yet (15 years later)
\begin{itemize}
\item Automatically realise ideal process models along selected innovation
  pathways. An automated “To be” process generator
\item P-TRIZ directions, linked to P-TRIZ operators, traverse an “As is”
  process model to generate a future process (increased Ideality)
\end{itemize}

\section{What is Innovation?}

Smith, Part 1:

\textbf{Innovation} is the process by which new commercial concepts –
products, services, processes – are brought into being, in order to generate
business. It requires uncontrollable creativity positioned side-by-side with
disciplined business practice.

Most companies find it tremendously difficult. Innovation, the goal of
creating new top-line value, is the antithesis of unreliable, hit-and-miss,
trial-and-error, psychological means of lateral thinking and brainstorming.
Rather, [...] innovation must be repeatable, procedural, and algorithmic.
Making effective progress requires much more than inspiration.

\subsection{Innovation as Challenge}
Innovation is about product \emph{and} processes of creation and use of the
product (see also the VDI norm 3780 defining the term \emph{technology}).

Smith, Part 3:

Taking a creative or innovative idea and turning it into cash involves almost
every part of a company. The new competitive battlefield is [...] the design,
the warranty, the service deal, the image, and the finance package.  [...] You
can hardly separate the product from the service, and all services are driven
by \emph{processes}.
  
Innovation is not only limited to high tech sectors of the economy, but is
rather an omnipresent driver for growth. Companies that recognise this will
not define innovation as owned by one part of the organisation or applying
only to those working in leading edge R\&D.  [...] These process innovations
are echoes of the reengineering mantra of the early 1990s.

The challenge in innovation today is thinking about and managing this
extremely broad set of interrelated activities as a unified process.

\subsection{Tool Based BPM}
APQC Cross Industry Process Classification Framework. Level 1 -- Categories
\cite{APQC-1,APQC-2,APQC-3}
\begin{itemize}
\item[1.0] Develop Vision and Strategy
\item[2.0] Develop and Manage Products and Services
\item[3.0] Market and Sell Products and Services 
\item[4.0] Deliver Physical Products 
\item[5.0] Deliver Services 
\item[6.0] Manage Customer Service 
\item[7.0] Develop and Manage Human Capital 
\item[8.0] Manage Information Technology (IT) 
\item[9.0] Manage Financial Resources 
\item[10.0] Acquire, Construct, and Manage Assets 
\item[11.0] Manage Enterprise Risk, Compliance, Remediation, and Resiliency 
\item[12.0] Manage External Relationships 
\item[13.0] Develop and Manage Business Capabilities
\end{itemize}

Smith, Part 3:

In the Third Wave of BPM, creative process design has been given a new path to
execution in the form of \textbf{business process management (BPM)
  systems}. These are IT tools that bring work processes to life in the
enterprise.

Such tools have transformed our ability to visualize, develop, and deploy
enterprise applications for much needed processes. BPM software provides many
benefits to both process owners and to IT developers [...] One documented
benefit is a reduced process design to deployment time and resource cost.

Yet BPM tools are no panacea. BPM deployment tools can only provide a fast
track to results once the process has been re-designed. Re-designing any
process beyond minor optimisation is still very much a creative human act.

\subsection{Innovation as Process}
Innovation as process re-design is itself a process that should be
systematically planned, designed and implemented.

Those who model, re-design, and deploy significant new business processes in
support of innovation also need a process. I call that process P-TRIZ.  [...]
The potential for a reliable and general-purpose innovation methodology that
can be applied to processes has never been greater.

\subsection{Innovation is Much More Than Invention}

Smith, Part 3:

\textbf{Innovation} is the end-to-end process by which improved, renewed, or
replacement products, solutions, and services are delivered in practice,
generating new “top line” business value.

Everyone has his or her own definition of innovation. It is now generally
agreed that \textbf{innovation is distinct from creativity and invention}, and
that it is an end-to-end process whose objective is the generation of value.

Companies sorely need creativity and talent, but they also need more than
bright ideas when reengineering. Dreaming up a process on a whiteboard is one
thing, making it happen quite another.

Making effective progress requires more than inspiration, it requires a
method. For too long, BPM has been an art. If results from BPM projects are
overly dependent on expensive consultants or rare insightful managers, BPM
will never cross-the-chasm and take its place among side more established
business practice.

Far from a sporadic inventive act, leading organisations treat innovation as a
\textbf{systemic and systematic activity}. That process can, within limits, be
codified and improved.

P-TRIZ is making its contribution in the domain of process design. It is part
of the journey towards a comprehensive process engineering that is less
reliant on unreliable, sporadic, and ad-hoc creativity.

My belief is that the next frontier for those wishing to advance the field of
BPM is where they will put themselves out of a job to \textbf{turn BPM into a
  science}, and deliver it in a form that can be used by everyman.

Where BPM put the engineering back into the IT-side of reengineering, P-TRIZ
seeks to \textbf{inject engineering into the creativity side of BPR}.

My vision for P-TRIZ is this: Where a BPMS can free business users from
dependence on technicians for the IT change around process, P-TRIZ will free
them from dependence on specialists for the process re-imagining around
change.

\section{What does TRIZ offer?}

Smith, Part 2: Process Innovation Models based on modelled Business Process
Landscapes with emphasis on
\begin{itemize}
\item distinguishing useful and harmful processes,
\item stronger cause-effect relations between functions, distinguishing
  relations producing or counteracting useful or harmful functions.
\end{itemize}
\emph{Abstract example:} 
\begin{center}
  \includegraphics[width=.5\textwidth]{images/SmithHoward-1.png}
\end{center}
Smith, Part 3: 
\begin{itemize}
\item Development of useful-harmful \emph{functional models} of processes
\item Identification of \emph{process-contradictions}: conflicting function
  points that link problem-solving knowledge to pain-points
\item Generation of \emph{(abstract) solution pathways} (BPR) 
\item Management of a "world-wide" solution knowledge for \emph{process
  re-design patterns}
\item Opening a path to \emph{process "Ideality"}.
\end{itemize}

P-TRIZ lets practitioners capture the “why” aspects of solution design in
addition to the “what/how” inherent to today’s BPM modeling paradigm. P-TRIZ
will be of interest to process and org-change practitioners who have never
modeled a business process formally in their life.

\subsection{Ideal Process}

If business leaders are to have full confidence in BPM methods and tools,
process \emph{innovation} must guarantee a progression towards \emph{more
  ideal processes}. This is what P-TRIZ seeks to achieve.

This is a general pattern for technological systems. They tend to evolve in
the direction of increasing ideality. In other words, systems become smaller,
less costly, more energy efficient, pollute less, and so on.

Processes are also technological systems. They comprise many participants –
people, systems, and machines – and they operate much like a complex machine
with many concurrently executing parts. They too will tend towards ideality.
If your competitors offer more ideal processes, your company is at a
disadvantage. All companies should be on the look out for more ideal
processes: That is what \emph{process innovation} means.

To get a measure of ideality, TRIZ proposes the ratio of a system’s useful
functions to its harmful functions.

\begin{gather*}
  \text{Ideality}=\frac{\text{Sum of useful functions}}{\text{Sum of harmful
      functions}} \tag{Smith' definition}
\end{gather*}

\begin{gather*}
  \text{Ideality}=\frac{\left(\parbox{3cm}{\centering Revenue from useful
      functions\vskip4pt}\right) - \left(\parbox{3cm}{\centering Losses from
      harmful functions\vskip4pt}\right)}{\text{Costs}} \tag{Souchkov's
    definition}
\end{gather*}
Smith: \emph{Ideality} is clearly a qualitative assessment rather than a
quantifiable number. Nonetheless, the concept of ideality is important, as it
helps us understand what reengineering must achieve – a more ideal system or
process.  It drives us to think about the process of process re-design so as
to achieve ideality.

For processes, \emph{functions} (useful or harmful) are their outputs,
activities, actions, steps, resources, tasks, or any other factors inherent to
the execution of the process.

The useful functions can be classified as follows:
\begin{itemize}
\item \emph{Primary (Main) useful function} – the purpose for which the
  process was designed.
\item \emph{Secondary functions} – other useful functions that the process
  provides in addition to the primary useful function.
\item \emph{Auxiliary functions} – functions that support or contribute to the
  execution of the primary function, such as corrective functions, control
  functions, compliance functions, etc.
\end{itemize}

\subsection{Process Vanishes -- the Ideal Machine}

Given the definition of ideality as the ratio between a system’s useful
functions and its harmful functions, we can imagine the most ideal system of
all. It would be a system in which there are no harmful functions at all – in
other words, it would cost nothing to design, implement, or maintain, use no
energy, take up no space, would emit no harmful byproducts, and so on. Stated
another way: \textbf{An ideal system is one whose functions are performed
  without the system existing; no “system” at all, just all the benefits.}

Taking this TRIZ principle and applying it processes, we have this: The
objective of reengineering is to \textbf{get rid of processes altogether!}

Wouldn’t we like all processes to be this way? Wouldn’t it be great if product
availability could be achieved without inventory? Wouldn’t you like to consume
services at zero cost to you? Shouldn’t a supply chain operate without a
supply chain process?

We never actually need a process; what we really need is a \emph{function}.
While this statement may sound strange, it is undoubtedly true. The objective
of reengineering is to turn processes into functions, and to remove activity,
leaving benefit.

Ideality – sometimes called the Ideal Final Result in TRIZ – is part of a
collective wisdom. How often has it been said that business processes are best
designed to provide their benefits in the \emph{simplest} possible manner? Now
we know what that means. We aspire to take the process out of the process.
P-TRIZ is bringing a science and approach to that common sense wisdom.

\subsection{Harmful and Useful}

Is Anything Ever Completely Useful Or Harmful?

As process engineers, we work to reveal, and then eradicate, the harmful
functions. We convert “As Is” process designs toward “To Be” process designs
by transforming the cause-effect links between useful and harmful elements,
and by finding solutions (new functions) that convert harm into useful output.

We also limit or counter-act the effect of harmful functions by exploiting
many kinds of available resources within or surrounding the domain of the
process and its environment – including relationships, time, finance, and many
other types of resources.

\textbf{Resource analysis plays a great role in P-TRIZ.}

\section{Why Using P-TRIZ (i.e. Business TRIZ?}

Companies using TRIZ find that it \emph{focuses} their knowledge and talents
on the problem-solving process. Specifically, TRIZ focuses on problems
preventing progress in innovation, the process by which new commercial
concepts – solutions, services, processes – are brought into being in order to
generate business. In this idea-to-cash process, a complex cocktail of
obstacles limits a company’s ability to innovate right across the value chain,
from mind-to-market, and covering every conceivable technical and managerial
discipline.

Examining the details of many TRIZ stories, Computer Sciences Corporation
concluded that TRIZ principles could be used in any field, not just
engineering. TRIZ is simple enough to be used in response to an email enquiry
and sophisticated enough to guide an entire program of activities.  TRIZ is
for school children as well as postgraduate scientists.

P-TRIZ can be considered an application of modern TRIZ.  P-TRIZ will add to
the body of worldwide TRIZ knowledge, including
\begin{itemize}
\item \emph{Specific vocabularies} for a consistent modelling of processes
  using TRIZ.
\item \emph{TRIZ solution patterns} that apply specifically to processes. 
\item \emph{Bindings} between TRIZ modeling constructs and accepted process
  modeling in languages and notations.
\item \emph{Evolutionary trends} observed as processes tend towards Ideality.
\item Workshop and project \emph{practices} that facilitate the practical and
  efficient use of TRIZ in a “commerce time” reengineering context.
\item A small number of \emph{extensions to the standard modern TRIZ
  notation}.  The objective is to enrich TRIZ formulation in support of
  Business Process and Enterprise Architecture Innovation.
\end{itemize}

Smith: We have been doing BPR for years. Most – but, likely, not all – of the
possible solution patterns are known. Now, with the advent of a BPMS that can
speed new processes to implementation, it would be foolish indeed to wait for
the right process expert to come along and help our improvement project.
Companies need a “just in time” process knowledge. It’s high time we encoded
reengineering wisdom and set out to create actionable insights for BPM
practitioners. I know of no better approach than TRIZ.

\begin{thebibliography}{xxx}
\bibitem{APQC-1} The APQC’s Process Classification Framework (PCF).\\
  \url{https://www.apqc.org/process-frameworks}
\bibitem{APQC-2} APCQ (2018). Cross Industry Process Classification Framework
  v.7.2.1
\bibitem{APQC-3} APCQ (2021). Best Practices in Applying Process Frameworks.
\bibitem{Smith-1} Howard Smith (2004). What Innovation Is – How Companies
  Develop Operating Systems For Innovation. Computer Sciences Corporation. 
\bibitem{Smith-2} Howard Smith (2005). From CIO to CPO via BPM: The Next
  Generation of Enterprise Automation. Computer Sciences Corporation.
\bibitem{Smith-3} Howard Smith (2006). P-TRIZ in the History of Business
  Process. Part 3 in a series on P-TRIZ.  Computer Sciences Corporation.
\end{thebibliography}

\end{document}

