\documentclass{beamer}
\usepackage{lsfolien}
\usepackage[english]{babel}

\myfootline{Sustainability, Environment, Management -- Summer Term
  2022}{Hans-Gert Gr\"abe}

\title{On the Notion of a Resource. Part 2\vskip1em}

\subtitle{Research Seminar in the Module 10-202-2312\\ for Master Computer
  Science}

\author{Prof. Dr. Hans-Gert Gräbe\\
\url{http://www.informatik.uni-leipzig.de/~graebe}}

\date{May 2022}
\begin{document}

{\setbeamertemplate{footline}{}
\begin{frame}
  \titlepage
\end{frame}}

\begin{frame}{Resources, Tools and State Changes} 

We worked out an \textbf{conceptional asymmetry between tool and workpiece}:
The tool (as a component within the system) provides a \emph{main useful
  function} that is applied to change the state of the workpiece. The state of
the tool remains (at least conceptually) unchanged.

In SF-modelling this asymmetry moves into the background, but the state-change
is rather matter of the \textbf{action} (as „pure functionality“ of the Ideal
Machine -- the machine disappeares) and no more the tool.

This matches the component concept as proposed by C. Szyperski for Component
Software.

\end{frame}

\begin{frame}{Components and Objects according to Szyperski}
  
\emph{Components} (which are „for composition“) are conceptualised
\begin{itemize}
\item as unit of independent deployment,
\item as unit of third party composition
\item having no (externally) observable state.
\end{itemize}
In contrast to this \emph{Objects} are conceptualised
\begin{itemize}
\item as unit of instantiation,
\item that may have externally observable states
\item and encapsulates its state and behaviour. 
\end{itemize}
Instantiation is important to maintain a certain standardisation of workpieces
required for a repeated application of a function within a production process.
\end{frame}

\begin{frame}{The World of Technical Systems}

The operational demand of a technical system is formulated in the form of
\emph{specifications} as requirements to the "environment", which must be
fulfilled for the \emph{operation} of the system. Thus the "reduction to the
essentials" that characterises the systemic approach is only a
\emph{conditional} mind game that presupposes a sufficiently powerful
\emph{environment} as given, in which the necessary \emph{resources} can be
found to fulfil the operating conditions.

Sommerville emphasises the importance of such interface specification for the
development of software systems that "need to interoperate with other systems
that have already been developed and installed in the environment."

\end{frame}

\begin{frame}{Components as Resources and Component Models}

The same perspective is significant when large systems are to be created in a
cooperative development process and for this a decomposition into subsystems
is required that are to be developed independently of each other.

This development process in turn requires a more extensive socio-technical
infrastructure with
\begin{enumerate}
\item \emph{independent components} that can be fully configured via their
  interfaces,
\item \emph{standards for components} that simplify their integration, 
\item a \emph{middleware}, which supports the component integration with
  software
\item and a \emph{development process} that is designed for component-based
  software engineering.
\end{enumerate}

\end{frame}

\begin{frame}{Components as Resources and Component Models}

Components are thus conceptually integrated into an overarching
\emph{component model}, which essentially ensures the technical
interoperability of different components beyond concrete interface
specifications and thus forms a moment of unity in the diversity of the
components.

However, this unity extends not only to the model, but also to the operating
conditions of the components (as functional property of the middleware) as
well as to their socio-technical development conditions (as a partial
formalisation of the development process).

This frame constitutes as \emph{component framework} (Szyperski) a
socio-technical supersystem as an "environment" of components that were
created according to the specifications of that component model.
  
\end{frame}

\begin{frame}{The World of Component Models}

Szyperski, for his part, analyses this diversity of compatibilities and
incompatibilities of different component models and identifies different
levels of abstraction for the reuse of concepts that go beyond the use of
prefabricated components.

In his 20-year-old book he already emphasises
\begin{quote}\small
  the growing importance of component deployment, and the relationship between
  components and services, the distinction of deployable components (or just
  components) from deployed components (and, where important, the latter again
  from installed components). Component instances are always the result of
  instantiating an installed component -- even if installed on the fly.
  Services are different from components in that they require a service
  provider.  
\end{quote}
  
\end{frame}

\begin{frame}{Functional and Attributive Properties}

Szyperski shows that the component approach is an approach of reuse that is
not limited to the (possibly modified) abstract reuse of the technical
functionality of a problem solution, but always reuses components together
with their operating conditions as \emph{services} and thus not detached from
their environment.

For this, Shchedrovitsky's distinction between functional and attributive
properties as well as the distinction between the notions of \emph{part} and
\emph{element} are essential.  
\begin{quote}\small
  {\bf Elements} are what a unity is made up of, so an element is a part
  inside the whole, which functions inside the unity, without as it were being
  torn out of it. A simple body, a {\bf part}, is what we have when everything
  has been disassembled and is laid out separately. But elements only exist
  within the structure of {\bf connections}. So an element implies two
  principally different types of properties: its properties as material, and
  its functional property derived from connections.
\end{quote}
\end{frame}

\begin{frame}{Functional and Attributive Properties}
\begin{quote}\small
  In other words, an element is not a part. A part exists when we mechanically
  divide something up, so that each part exists on its own as a simple body.
  An element is what exists in connections within the structure of the whole
  and functions there. [\ldots]\vskip1em

  {\bf Functional properties} belong to an element to the extent that it
  belongs to the structure with connections, while other properties belong to
  the element itself. If I take out this piece of material, it preserves its
  {\bf attributive properties}. They do not depend on whether I take it out of
  the system or put it into the system. But functional properties depend on
  whether or not there are connections. They belong to the element, but they
  are created by a connection; they are brought to the element by connections.
\end{quote}

\end{frame}

\begin{frame}{Filling the Places with Content}
The terms \emph{part}, \emph{element}, \emph{connection} describe the
\emph{structure} of the place in the system itself, where the connection of
the "dead" system with the "living" world must be carried out in order to
bring the system itself to life.

In the further system genesis, this conceptual frame has to be filled with
suitable resources. How conceptualise this "filling“, the combination of the
functional properties at the "connections" with resources to an almost
ideal machine?

To describe this composition process ("components are for composition" --
Szyperski) Shchedrovitsky distinguishes the concepts \emph{place} and
\emph{content}.

\end{frame}

\begin{frame}{Filling the Places with Content}
\begin{quote}\small
  Doing that, we introduce the concepts of place and content. An {\bf element}
  is a unity of a place and its content – the unity of a functional place, or
  a place in the structure, and what fills this place.\vskip1em

  A {\bf place} is something that possesses functional properties. If we take
  away the content, take it out of the structure, the place will remain in the
  structure, held there by {\bf connections}. The place bears the totality of
  functional properties.\vskip1em

  The {\bf content} by contrast is something that has attributive functions.
  Attributive functions are those that are retained by the content of a place,
  when this content is taken out of the given structure. We never know whether
  these are its properties from another system or not. Now we might take
  something out as content, but it is in fact tied to another system, which,
  as it were, extends through this place. 
\end{quote}
\end{frame}

\begin{frame}{Filling the Places with Content}
The search for resources is constitutive for the process of confinement in the
course of the genesis of the system that is to be developed from the pure
functionality of the ideal machine.  This corresponds to Altshuller's first
law of systemic development.
\begin{block}{Altshuller's Law of the Completeness of the Parts of a System}
  The necessary condition for the viability of a technical system is the
  existence of the main parts of the system {\bf and} their minimal
  functionality (i.e. viability -- HGG).
\end{block}
\vfill

However, the thing viewed with the magnifying glass as a connection of place
and content remains a "dead body", because "a living being has no parts"
(Shchedrovitsky).

\end{frame}

\begin{frame}{Connecting Systems. The Operational Dimension}
It is of little use to dissect a living frog in order to see how place and
content are to be combined, since you cannot study the blood flow in its veins
this way.

It is not enough that the plug fits into the socket, the socket must also
"have power in it". 

Beyond the connection of place and content an operational process dimension is
essential for a living system.  Shchedrovitsky develops that as a \emph{second
  concept of a system}.  This cannot be explained here.

We are dealing with a typical phenomenon of a modern society, in which the
electricity comes from the socket and the milk from the shop.  The division of
labour in such a modern mode of production leads to the emergent phenomenon of
social unity and stratification of the reproduction of infrastructural
conditions.

\end{frame}

\begin{frame}{Connecting Systems. The Operational Dimension}

The existence, reliability and robustness (resilience) of such an
infrastructure has a significant influence on the way people organise their
daily lives.

With the insight into ever more complex interrelationships, a concept of
resources as "anything in or around the system that is not being used to its
maximum potential" (Mann, Salamatov), which focuses on the \emph{exploitation}
of resources, becomes increasingly counterproductive and has to be replaced by
a concept of resources with socio-culturally institutionalised forms of
\emph{resource management} at its center.

\end{frame}

\begin{frame}{The Concept of a Resource and the Mode of Production}

\begin{block}{Thesis:}
  The concept of resource exploitation is a characteristic feature of all
  existing so far forms of a capitalist mode of production.
\end{block}

It manifests a fundamental contradiction of socio-cultural development:
without such exploitation we would not have reached the current state of
technology, but at the same time we undermine our own conditions of existence.

My historical optimism says that it is nevertheless precisely these means of
increasing conceptual penetration of ever increasingly complex
interrelationships by which this trend can be stopped and eventually reversed.

\end{frame}

\begin{frame}{The Global Scope of Local Action}
The formulated contradiction is of a global, planetary dimension that cannot
be solved by the regional disposition of individual power groups over
exploitable resources.  The division of the world into spheres of influence
thus becomes obsolete insofar as in each of these spheres of influence, the
transition to a different form of using resources must be organised to avoid a
global environmental collapse of the resources used by mankind in the long
run.

TRIZ systemic evolution trends of increasing coordination, controllability and
dynamisation refer not only to \emph{system-internal development lines}, but
also to the coordination \emph{between} systems which are operated by
independent third parties.

Qualifying the infrastructural framework, for example, of the power supply
system as "supersystem" does not take into account the relations of
\emph{mutual interdependency} in such a modern industrial mode of production.

\end{frame}

\end{document}
