\documentclass{beamer}
\usepackage{lsfolien}
\usepackage[english]{babel}

\myfootline{Sustainability, Environment, Management -- Summer Term
  2022}{Hans-Gert Gr\"abe}

\title{P-TRIZ in the History of Business Processes\vskip1em}

\subtitle{Research Seminar in the Module 10-202-2312\\ for Master Computer
  Science}

\author{Prof. Dr. Hans-Gert Gräbe\\
\url{http://www.informatik.uni-leipzig.de/~graebe}}

\date{May 10, 2022}
\begin{document}

{\setbeamertemplate{footline}{}
\begin{frame}
  \titlepage
\end{frame}}

\begin{frame}{Development of Business Process Modelling (BPM)} 

\begin{itemize}
\item 1990: Business Process Reengineering (BPR)
\item 2000: Business Process Management Systems (BPMS)
\item 2005: Business Process Innovation (P-TRIZ)
\item 200?: Business Process Realisation (Transformation)
\end{itemize}

Each era of BPM has added new capabilities to the last.

BPMS systems enable process architects to readily deploy creative new process
designs, sidestepping time and resource intensive implementation projects of
the past that so denuded and distorted reengineering of its creative
potential.

Now, P-TRIZ is an emerging method that builds on the shoulders of those
giants.

\end{frame}

\begin{frame}{What is Innovation?}

\textbf{Innovation} is the process by which new commercial concepts –
products, services, processes – are brought into being, in order to generate
business. It requires uncontrollable creativity positioned side-by-side with
disciplined business practice.

The challenge in innovation today is thinking about and managing this
extremely broad set of interrelated activities as a unified process.
\end{frame}

\begin{frame}{Tool Based BPMS}
The developments since the 1990s have led to a very detailed understanding of
how company-internal processes are to be structured and which areas are to be
differentiated.  Corresponding cross-industry frameworks have become
established, such as the seven levels of APQC.

In the Third Wave of BPM, creative process design has been given a new path to
execution in the form of \textbf{business process management (BPM)
  systems}. These are IT tools that bring work processes to life in the
enterprise.

Yet BPM tools are no panacea. BPM deployment tools can only provide a fast
track to results once the process has been re-designed. Re-designing any
process beyond minor optimisation is still very much a creative human act.
\end{frame}

\begin{frame}{Innovation as Process}

Innovation as process re-design is itself a process that should be
systematically planned, designed and implemented.

Smith: Those who model, re-design, and deploy significant new business
processes in support of innovation also need a process. I call that process
P-TRIZ.  [...]  The potential for a reliable and general-purpose innovation
methodology that can be applied to processes has never been greater.
\end{frame}

\begin{frame}{Innovation and Invention}

\textbf{Innovation} is the end-to-end process by which improved, renewed, or
replacement products, solutions, and services are delivered in practice,
generating new “top line” business value.

It is now generally agreed that \textbf{innovation is distinct from creativity
  and invention}, and that it is an end-to-end process whose objective is the
generation of value.

Making effective progress requires more than inspiration, it requires a
method. 
\end{frame}

\begin{frame}{What does TRIZ offer?}

A method for organising the innovation process. What does TRIZ offer?
\begin{itemize}
\item Development of useful-harmful \emph{functional models} of processes
\item Identification of \emph{process-contradictions}: conflicting function
  points that link problem-solving knowledge to pain-points
\item Generation of \emph{(abstract) solution pathways} (BPR) 
\item Management of a "world-wide" solution knowledge for \emph{process
  re-design patterns}
\item Opening a path to \emph{process "Ideality"}.
\end{itemize}
\end{frame}

\begin{frame}{An Abstract Example}
\begin{center}
  \includegraphics[width=.8\textwidth]{images/SmithHoward-1.png}
\end{center}
\end{frame}

\begin{frame}{The Ideal Process}

If business leaders are to have full confidence in BPM methods and tools,
process \emph{innovation} must guarantee a progression towards \emph{more
  ideal processes}. This is what P-TRIZ seeks to achieve.

If your competitors offer more ideal processes, your company is at a
disadvantage. All companies should be on the look out for more ideal
processes: That is what \emph{process innovation} means.
\end{frame}

\begin{frame}{The Ideal Process}
To get a measure of ideality, TRIZ proposes the ratio of a system’s useful
functions to its harmful functions.

\begin{gather*}
  \text{Ideality}=\frac{\text{Sum of useful functions}}{\text{Sum of harmful
      functions}} \tag{Smith' definition}
\end{gather*}

\begin{gather*}
  \text{Ideality}=\frac{\left(\parbox{3cm}{\centering Revenue from useful
      functions\vskip4pt}\right) - \left(\parbox{3cm}{\centering Losses from
      harmful functions\vskip4pt}\right)}{\text{Costs}} \tag{Souchkov's
    definition}
\end{gather*}
\end{frame}

\begin{frame}{Useful and Harmful Functions}

For processes, \emph{functions} (useful or harmful) are their outputs,
activities, actions, steps, resources, tasks, or any other factors inherent to
the execution of the process.

The useful functions can be classified as follows:
\begin{itemize}
\item \emph{Primary (Main) useful function} – the purpose for which the
  process was designed.
\item \emph{Secondary functions} – other useful functions that the process
  provides in addition to the primary useful function.
\item \emph{Auxiliary functions} – functions that support or contribute to the
  execution of the primary function, such as corrective functions, control
  functions, compliance functions, etc.
\end{itemize}
\end{frame}

\begin{frame}{Processes Vanish -- the Ideal Machine}
Given the definition of ideality as the ratio between a system’s useful
functions and its harmful functions, we can imagine the most ideal system of
all.

It would be a system in which there are no harmful functions at all – in other
words, it would cost nothing to design, implement, or maintain, use no energy,
take up no space, would emit no harmful byproducts, and so on.

Stated another way: \textbf{An ideal system is one whose functions are
  performed without the system existing; no “system” at all, just all the
  benefits.}

\end{frame}

\begin{frame}{Processes Vanish -- the Ideal Machine}
Taking this TRIZ principle and applying it processes, we have this: The
objective of reengineering is to \textbf{get rid of processes altogether!}

We never actually need a process; what we really need is a \emph{function}.
While this statement may sound strange, it is undoubtedly true. The objective
of reengineering is to turn processes into functions, and to remove activity,
leaving benefit.
\end{frame}

\begin{frame}{Systemic Development}


As process engineers, we work to reveal, and then eradicate, the harmful
functions. We convert “As Is” process designs toward “To Be” process designs
by transforming the cause-effect links between useful and harmful elements,
and by finding solutions (new functions) that convert harm into useful output.

We also limit or counter-act the effect of harmful functions by exploiting
many kinds of available resources within or surrounding the domain of the
process and its environment – including relationships, time, finance, and many
other types of resources.

\textbf{Resource analysis plays a great role in P-TRIZ.}

\end{frame}

\begin{frame}{Why Using P-TRIZ (i.e. Business TRIZ)?}\small
P-TRIZ can be considered an application of modern TRIZ.  P-TRIZ will add to
the body of worldwide TRIZ knowledge, including
\begin{itemize}
\item \emph{Specific vocabularies} for a consistent modelling of processes
  using TRIZ.
\item \emph{TRIZ solution patterns} that apply specifically to processes. 
\item \emph{Bindings} between TRIZ modeling constructs and accepted process
  modeling in languages and notations.
\item \emph{Evolutionary trends} observed as processes tend towards Ideality.
\item Workshop and project \emph{practices} that facilitate the practical and
  efficient use of TRIZ in a “commerce time” reengineering context.
\item A small number of \emph{extensions to the standard modern TRIZ
  notation}.  The objective is to enrich TRIZ formulation in support of
  Business Process and Enterprise Architecture Innovation.
\end{itemize}
\end{frame}

\end{document}
